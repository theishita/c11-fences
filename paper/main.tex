% This is samplepaper.tex, a sample chapter demonstrating the
% LLNCS macro package for Springer Computer Science proceedings;
% Version 2.20 of 2017/10/04
%
\documentclass[runningheads]{llncs2e/llncs}
%
\usepackage{graphicx}
\usepackage{xspace}
\usepackage[dvipsnames]{xcolor}
\usepackage{amsmath}
\usepackage{afterpage}  
\usepackage{figlatex,wrapfig}
\usepackage[dvipsnames]{xcolor}
\usepackage{listings,amssymb,mathtools}
\usepackage{mathrsfs}
\usepackage{array,multirow}
\usepackage{caption}
\usepackage{longtable}
\usepackage{algorithm}
\usepackage[noend]{algpseudocode}
\usepackage{framed,enumitem}
\usepackage{wrapfig}

% for colours
\usepackage{xspace}
\usepackage[colorlinks]{hyperref}
\hypersetup{
	colorlinks = true,
	citecolor = {Magenta},
	linkcolor = {blue},
	urlcolor  = {blue}
}

% for arrow diagrams
\usepackage{amsmath}
\usepackage{amssymb}
\usepackage{smartdiagram}
\usepackage{tikz}
\usetikzlibrary{arrows,positioning}

%\usepackage[parfill]{parskip}

% for bib handline
%\usepackage[numbers]{natbib}
%\usepackage{url}

% implies and iff arrows
\renewcommand{\iff}{\xspace\Leftrightarrow\xspace}
\renewcommand{\implies}{\xspace\Rightarrow\xspace}
\newcommand{\onlyif}{\xspace\Leftarrow\xspace}

% commonly used abbreviations and expressions
\renewcommand{\th}{^{th}\xspace} % superscript th for numbers eg i^{th}
\newcommand{\definedas}{\triangleq\xspace}
\newcommand{\ie}{{\em i.e.}\xspace}
\newcommand{\st}{\ \mbox{s.t.}\ }
\newcommand{\viz}{\textit{viz}.\@\xspace}
\newcommand{\wrt}{\textit{wrt}\xspace}
\newcommand{\wkt}{we know that,\xspace}
\newcommand{\aka}{a.k.a\xspace}
\newcommand{\sota}{state-of-the-art\xspace}
\newcommand{\Sota}{State-of-the-art\xspace}

% common operators
\renewcommand{\^}{\xspace\wedge\xspace}
\renewcommand{\v}{\xspace\vee\xspace}
\newcommand{\xor}{\xspace\veebar\xspace}
\renewcommand{\|}{\ |\ }
\newcommand{\intersection}{\xspace\cap\xspace}
\newcommand{\union}{\xspace\cup\xspace}
\newcommand{\intersectioneq}{\xspace\cap=\xspace}
\newcommand{\unioneq}{\ {\cup}{=}\ }
\newcommand{\nin}{\not\in\xspace}

% highlighted hyperlinks
\newcommand{\hl}[1]{{\textcolor{darkgray}{\texttt{(#1)}}}\xspace} % hyperlink target
\newcommand{\hlref}[1]{\hyperlink{#1}{\textcolor{Sepia}{\small \texttt{(#1)}}}}
\newcommand{\tab}{\quad\quad}

%%%%%%%%%%%%%%%%%%%%%%%%% Document Specific %%%%%%%%%%%%%%%%%%%%%%

% tools and techniques
\newcommand{\ourtechnique}{\textcolor{RubineRed}{\texttt{FenSying}}\xspace}
\newcommand{\ourtool}{\ourtechnique{-}tool\xspace}
\newcommand{\cc}{\textit{C11}\xspace}
\newcommand{\cds}{CDSChecker\xspace}
\newcommand{\genmc}{GenMC\xspace}
\newcommand{\tracer}{Tracer\xspace}
\newcommand{\z}{\texttt{Z3}\xspace}

% sets and entitites
\newcommand{\program}{$P$\xspace} %input program
\newcommand{\programhat}{$\widehat{P}$\xspace} %transformed/fixed program
\newcommand{\inv}[1]{{#1}^{\mathtt{inv}}\xspace}
\newcommand{\imm}[1]{{#1}^{\mathtt{imm}}\xspace}
\newcommand{\fx}[1]{{#1}^{\mathtt{fx}}\xspace}
\newcommand{\formula}[1]{\mathscr{F}(#1)\xspace}
\newcommand{\threads}{\mathcal{T}\xspace}
\newcommand{\states}{\Sigma\xspace}
\newcommand{\moset}{\mathcal{M}\xspace}
\newcommand{\actions}{\mathcal{A}\xspace}
\newcommand{\objects}{\mathcal{O}\xspace}
\newcommand{\s}[1]{s_{[#1]}\xspace} % state reached after exploring sequence #1
% events' sets aux
\newcommand{\wt}[1]{{\mathbb{W}#1}}
\newcommand{\rd}[1]{{\mathbb{R}#1}}
\newcommand{\fn}[1]{{\mathbb{F}#1}}
% events' sets
\newcommand{\events}{\mathcal{E}\xspace}
\newcommand{\writes}{\events^\wt{}\xspace}
\newcommand{\reads}{\events^\rd{}\xspace}
\newcommand{\fences}{\events^\fn{}\xspace}
\newcommand{\ordevents}[1]{\events^{(#1)}\xspace}
\newcommand{\ordwrites}[1]{\events^\wt{(#1)}\xspace}
\newcommand{\ordreads}[1]{\events^\rd{(#1)}\xspace}
\newcommand{\ordfences}[1]{\events^\fn{(#1)}\xspace}


% memory orders
\newcommand{\mosc}{\texttt{seq\_cst}\xspace}
\newcommand{\moar}{\texttt{acq\_rel}\xspace}
\newcommand{\morel}{\texttt{release}\xspace}
\newcommand{\moacq}{\texttt{acquire}\xspace}
\newcommand{\mocon}{\texttt{consume}\xspace}
\newcommand{\morlx}{\texttt{relaxed}\xspace}

% operators
\newcommand{\molt}{{\sqsubset}\xspace}
\newcommand{\mole}{{\sqsubseteq}\xspace}
\newcommand{\mogt}{{\sqsupset}\xspace}
\newcommand{\moge}{{\sqsupseteq}\xspace}

% relations
\newcommand{\reln}[4]{#3 {\rightarrow^{#1}_{#2}} #4\xspace} % any relation specified as #1
\newcommand{\nreln}[4]{#3 \nrightarrow^{#1}_{#2} #4\xspace} % not of any relation specified as #1

% relation with events
\newcommand{\seqb}[3]{\reln{\textbf{\textcolor{CarnationPink}{sb}}}{#1}{#2}{#3}\xspace}
\newcommand{\rf}[3]{\reln{\textbf{\textcolor{PineGreen}{rf}}}{#1}{#2}{#3}\xspace} 
\newcommand{\dob}[3]{\reln{\textbf{\textcolor{Mulberry}{dob}}}{#1}{#2}{#3}\xspace}
\newcommand{\sw}[3]{\reln{\textbf{\textcolor{Magenta}{sw}}}{#1}{#2}{#3}\xspace}
\newcommand{\ithb}[3]{\reln{\textbf{\textcolor{NavyBlue}{ithb}}}{#1}{#2}{#3}\xspace}
\newcommand{\hb}[3]{\reln{\textbf{\textcolor{Cerulean}{hb}}}{#1}{#2}{#3}\xspace}
\newcommand{\nhb}[3]{\nreln{\textbf{\textcolor{Cerulean}{hb}}}{#1}{#2}{#3}\xspace}
\newcommand{\mo}[3]{\reln{\textbf{\textcolor{RedOrange}{mo}}}{#1}{#2}{#3}\xspace}
\newcommand{\nmo}[3]{\nreln{\textbf{\textcolor{RedOrange}{mo}}}{#1}{#2}{#3}\xspace}
\renewcommand{\to}[3]{\reln{\textbf{\textcolor{Brown}{to}}}{#1}{#2}{#3}\xspace}
\newcommand{\so}[3]{\reln{\textbf{\textcolor{Mahogany}{so}}}{#1}{#2}{#3}\xspace} %sc
\newcommand{\fr}[3]{\reln{\textbf{\textcolor{RoyalPurple}{fr}}}{#1}{#2}{#3}} %rb

% relation name without events
\newcommand{\setSB}{\seqb{\tau}{}{}\xspace}
\newcommand{\setRF}{\rf{\tau}{}{}\xspace}
\newcommand{\setSW}{\sw{\tau}{}{}\xspace}
\newcommand{\setDOB}{\dob{\tau}{}{}\xspace}
\newcommand{\setITHB}{\ithb{\tau}{}{}\xspace}
\newcommand{\setHB}{\hb{\tau}{}{}\xspace}
\newcommand{\setMO}{\mo{\tau}{}{}\xspace}
\newcommand{\setTO}{\to{\tau}{}{}\xspace}
\newcommand{\setSO}{\so{\tau}{}{}\xspace}
\newcommand{\nsetHB}{\nhb{\tau}{}{}\xspace}
\newcommand{\nsetMO}{\nmo{\tau}{}{}\xspace}
\newcommand{\setFR}{\fr{\tau}{}{}\xspace}

% relation label 
\newcommand{\lsb}{\textbf{\textcolor{CarnationPink}{sb}}\xspace}
\newcommand{\lrf}{\textbf{\textcolor{PineGreen}{rf}}\xspace} 
\newcommand{\ldob}{\textbf{\textcolor{Mulberry}{dob}}\xspace}
\newcommand{\lsw}{\textbf{\textcolor{Magenta}{sw}}\xspace}
\newcommand{\lithb}{\textbf{\textcolor{NavyBlue}{ithb}}\xspace}
\newcommand{\lhb}{\textbf{\textcolor{Cerulean}{hb}}\xspace}
\newcommand{\lmo}{\textbf{\textcolor{RedOrange}{mo}}\xspace}
\newcommand{\lto}{\textbf{\textcolor{Brown}{to}}\xspace}
\newcommand{\lso}{\textbf{\textcolor{Mahogany}{so}}\xspace}
\newcommand{\lfr}{\textbf{\textcolor{RoyalPurple}{fr}}\xspace}

\newcommand{\var}[1]{\color{OliveGreen}\texttt{#1}\color{black}\xspace}
\newcommand{\fun}[2]{\color{Sepia}\texttt{#1(\color{Gray}\textit{#2}\color{Sepia})}\color{black}\xspace}
\newcommand{\class}[1]{\color{DarkOrchid}\texttt{#1}\color{black}\xspace}

% memory orders
\newcommand{\na}{\texttt{na}\xspace}
\newcommand{\rlx}{\texttt{rlx}\xspace}
\newcommand{\rel}{\texttt{rel}\xspace}
\newcommand{\acq}{\texttt{acq}\xspace}
\newcommand{\acqrel}{\texttt{acq-rel}\xspace}
\renewcommand{\sc}{\texttt{sc}\xspace}

% load/store events and instructions
\newcommand{\load}[3]{#1 := #2_{#3}}
\newcommand{\store}[3]{#1 := #2_{#3}}
\newcommand{\loadev}[3]{\mathtt{R^{#3}({#1},{#2})}}
\newcommand{\storeev}[3]{\mathtt{W^{#3}({#1},{#2})}}
\newcommand{\fenceev}[1]{\textcolor{Brown}{--\fn{#1}--}}
% tikz edges
\newcommand{\rfedge}[3]{\draw [->,>=stealth,color=PineGreen,thin] ({#1}) -- node[{#3}] { rf} ({#2});}
\newcommand{\moedge}[3]{\draw [->,>=stealth,color=RedOrange,thin] ({#1}) -- node[{#3}] {mo} ({#2});}

\newcommand{\cycles}[1]{\mathcal{C}_{#1}}


%snj: Have to use the ones in format
%\newtheorem{theorem}{Theorem}[section]
%\newtheorem{corollary}{Corollary}[theorem]
%\newtheorem{lemma}[theorem]{Lemma}

\newcommand{\ishComment}[1]{\textit{\color{red}\tiny{#1}}}
\newcommand{\divComment}[1]{\textcolor{ForestGreen}{[div: #1]}}
\newcommand{\snj}[1]{\textcolor{RubineRed}{[snj]: #1}}
\newcommand{\svs}[1]{\textcolor{Maroon}{[svs]:#1}}


%svs -- added to track changes -- for the benefit of snj,div,ish!  use
% the [final] option to clear the changes and show the last changes
% only.
\usepackage[commentmarkup=todo,highlightmarkup=background]{changes}


% Used for displaying a sample figure. If possible, figure files should
% be included in EPS format.
%
% If you use the hyperref package, please uncomment the following line
% to display URLs in blue roman font according to Springer's eBook style:
% \renewcommand\UrlFont{\color{blue}\rmfamily}

\begin{document}
%
\title{Optimal Fence Synthesis for C/C++11}
%
%\titlerunning{Abbreviated paper title}
% If the paper title is too long for the running head, you can set
% an abbreviated paper title here
%
\author{First Author\inst{1}\orcidID{0000-1111-2222-3333} \and
Second Author\inst{2,3}\orcidID{1111-2222-3333-4444} \and
Third Author\inst{3}\orcidID{2222--3333-4444-5555}}
%
\authorrunning{F. Author et al.}
% First names are abbreviated in the running head.
% If there are more than two authors, 'et al.' is used.
%
\institute{Princeton University, Princeton NJ 08544, USA \and
Springer Heidelberg, Tiergartenstr. 17, 69121 Heidelberg, Germany
\email{lncs@springer.com}\\
\url{http://www.springer.com/gp/computer-science/lncs} \and
ABC Institute, Rupert-Karls-University Heidelberg, Heidelberg, Germany\\
\email{\{abc,lncs\}@uni-heidelberg.de}}
%
\maketitle              % typeset the header of the contribution
%
\begin{abstract}
The abstract should briefly summarize the contents of the paper in
150--250 words.

\keywords{\cc  \and Fence Synthesis \and Another keyword.}
\end{abstract}
%
%
%
\section{Introduction} \label{sec:intro}

\section{Preliminaries} \label{sec:preliminaries}
Consider a\deleted{n acyclic} multi-threaded \cc
program \added{$P$ $:=$ $\parallel_{i\in \text{\tt TID}} P_i$, 
where $\text{\tt TID}= \{1,\ldots,n\}$ is the set of thread ids}. 
\deleted{The} \added{Each} thread \deleted {of the program}
\added{$P_i$ is a loop-free program, which}
performs a sequence of memory access operations on a set of shared
memory objects and \cc memory fences.  The memory access operations
can be atomic or non-atomic in nature.
%
An instance of a thread operation in an execution is called an {\em
event}.  Events of a thread $t$ are uniquely indexed with an id.
%
\begin{definition}[Event]\newline
An event $i$ of thread $t$ is represented by a tuple $\langle i, t, act, obj,$ $ ord, inst \rangle$ where:
\begin{itemize}[label=inst,align=left,leftmargin=*]
\item [$act$] represents the event action $\in \{ \text{\tt read}, \text{\tt write}, \text{\tt rmw}, \text{\tt fence} \} $,
\item [$obj$] is the set of memory objects accessed,
\item [$ord$] records the \cc memory order associated with the event, and
\item [$inst$] is the corresponding program instruction.
\end{itemize}
\end{definition}
The $act$ {\tt rmw} represents {\em read-modify-write}.
%
Note that the set of memory objects of an rmw event can be non-singleton 
and for a fence event it is an empty set.
%
Let $\events$ denote the set of all program events. Furthermore,
$\writes$, $\reads$ and $\fences$ denote the write, read and fence 
events of the input program.
%
Throughout the text, use of read event as well as write event includes rmw
events unless specified otherwise.
%
\begin{definition}[Trace]\newline
	A trace or a maximal execution (or simply execution) $\tau$ of the input 
	program $P$ under \cc is a tuple 
	$\langle \events_\tau, \setHB, \setMO, \setRF \rangle$, where
	\begin{itemize}[label=sethb,align=left,leftmargin=*]
		\item [$\events_\tau$] represents the set of events in the trace $\tau$,
		\item [$\setHB$] ({\em Happen-before} relation) is a partial order on
			$\events_\tau$ representing the event interactions and inter-thread
			synchronizations, discussed in Section~\ref{sec:c11},
		\item [$\setMO$] ({\em Modification-order}) is a total order on the
			writes of an object that establishes coherence of $\tau$ 
			\wrt $\setHB$ , and
		\item [$\setRF$] ({\em Reads-from}) is a relation from a write event to
			a read event signifying that the read event takes the value of 
			the write event in $\tau$.
	\end{itemize}
\end{definition}

\noindent
{\bf Memory ordering under \cc}: 
\deleted{Each memory} The memory access and fence operations
\deleted{in a \cc program} \deleted{is} \added{are} 
associated with \deleted{a memory} ordering \added{modes}
that defines the ordering restriction \added{ on them.}
\deleted{ placed on atomic and non-atomic access around atomic memory access.}
%
\deleted{An event shares the same memory order as its corresponding operation
in the input program.}
%
\deleted{\cc provides the following set of memory 
 orders} $\moset$ = $\{ \na, \rlx, \rel, \acq, \acqrel, \sc \}$, 
represents the orders relaxed (\rlx), release (\rel), acquire (\acq),
acquire-release (\acqrel) and sequentially consistent (\sc) for
atomic events. A non-atomic event has \na memory order associated with 
it.
%
We use $\ordevents{m}_\tau$ (and accordingly $\ordwrites{m}_\tau$, 
$\ordreads{m}_\tau$ and $\ordfences{m}_\tau$) to represent the $m$
ordered events of an execution sequence $\tau$ (where, $m \in \moset$);
for example $\ordwrites{\rel}_\tau$ represents the write events of 
$\tau$ with ordering restriction \rel.

\deleted{
\noindent
{\bf \cc fences}: \cc provides atomic thread fences or simply 
fences to provide additional reordering restrictions on program 
events. Note that \cc fences are not memory barriers and do not
provide support for flushing local write values to shared memory.
%
A fence can be associated with memory orders $\acqrel$ and $\sc$
providing varying degrees of reordering restrictions.}

\noindent
{\bf Buggy and invalidated executions}: The threads of 
program $P$ may contain {\em assert checks} as a means of providing
program specification. A trace that violates an 
assert check (\ie the condition in the assert check computes to
{\em false}) is called a buggy trace.

The purpose of this work is to synthesize \cc fences at appropriate
program locations to invalidate buggy traces. Particularly, the
event relation in the buggy traces with synthesized fences
render the resulting program behavior invalid under \cc, thus ensuring
that a previously buggy trace would not materialize as a 
\cc program execution.

We represent the invalidated or fixed trace corresponding to a 
buggy trace $\tau$ by $\fx{\tau}$. 
%
As an intermediate step between $\tau$ and $\fx{\tau}$, we form an 
intermediate version of the trace $\tau$ with candidate fences
some of which are retained as a part of $\fx{\tau}$. We represent
the intermediate version of $\tau$ as $\im{\tau}$. The details
of the intermediate step are discussed in Section~\ref{sec:methodology}.
%
We also use $\im{P}$ and $\fx{P}$ to represent the intermediate and
fixed versions of the input program $P$.

\section{Background: C11 Memory Model} \label{sec:c11}
The \cc memory model forms an irreflexive and acyclic relation over the events of 
an execution sequence $\tau$ called the {\em happens-before} relation 
($\setHB \subseteq \events_\tau {\times} \events_\tau$);
such that, the execution sequence is {\em coherent} (or valid) under \cc if 
the $\setHB$ relation does not violate any {\em coherence rule} of \cc.
%
The events of a sequence $\tau$ are related by the $\setHB$ as follows;
\begin{itemize}
	\item {\em Intra-thread-hb}: Events of a thread are related by a {\em sequenced-before}
	($\setSB$) relation in the order of occurrence in the thread;
	
	\item {\em Inter-thread-synchronization}: A $\moge$\acq read event ($ra$) of a thread 
	$t_i$ when reads from a $\moge$\rel write event ($wr$) of another thread $t_j$ they form a
	relation {\em synchronizes-with} ($\setSW$) as $\sw{\tau}{wr}{ra}$.
	%
	Further, a $\moge$\acq read event ($ra$) of a thread $t_i$ when reads from a write 
	event ($w$) in the {\em Release sequence} \cite{C11} of a $\moge$\rel write event ($wr$) of 
	another thread $t_j$ then $wr$ and $ra$ form a relation {\em dependency-ordered-before}
	($\setDOB$) between them as $\dob{\tau}{wr}{ra}$;
	%
	(where, a release sequence headed by a $\moge$\rel write event $wr$ is the maximal 
	contiguous sequence, of an execution sequence $\tau$, that starts with $wr$ and continues 
	over write events from the same thread and rmw events or $\moge$\rel write events from
	other threads.)
	
	\item {\em Inter-thread-fence-synchronization}: The relation $\setSW$ is also formed 
	between \cc fences when a read event $\setSB$ ordered before a fence $f_i$ of thread 
	$t_i$ reads from a write event $\setSB$ ordered after a fence $f_j$ in another thread
	$t_j$.
	%
	Similarly, a fence $\setSB$ ordered before a write can form $\setSW$ with a $\moge$\acq
	read and a fence $\setSB$ ordered after a read can form $\setSW$ with a $\moge$\rel
	write when the read reads-from the write.
	
	\item {\em Inter-thread-hb}: Events from different threads that are related by the 
	transitive closure of ($\setSB$ $\union$ $\setSW$ $\union$ $\setDOB$) are related by 
	the {\em inter-thread-hb} relation ($\setITHB$). 
\end{itemize}

Accordingly, two events $e_1,e_2$ in an execution sequence $\tau$ are happens-before 
related \ie
$\hb{\tau}{e_1}{e_2}$ if $\seqb{\tau}{e_1}{e_2}$ or $\ithb{\tau}{e_1}{e_2}$.
%
Additionally, in a sequence $\tau$, all write events of an object $x$ are related by a 
total order called {\em modification-order} ($\setMO$).
%
The $\setMO$ order is constructed in compliance to a set of coherence
\lmo-rules defined by \cc based on $\setHB$ and $\setRF$ ({\em reads-from}, 
relation between a write and a read that reads the value of the write event). 
\snj{I have removed the formal definitions of the \lmo-rules, they are there 
in the latex file as comment if needed.}
%
%The set of $\setMO$ rules are,
%\begin{itemize}[label=moWW,align=left,leftmargin=*]
%	\item [\hl{moWW}:] $\forall w_1, w_2 \in \writes_\tau$ if $\hb{\tau}{w_1}{w_2}$ then
%						$\mo{\tau}{w_1}{w_2}$\newline
%						($\setHB$ ordered writes are ordered by $\setMO$);
%	\item [\hl{moRR}:] $\forall r_1, r_2 \in \reads_\tau$ \st $\hb{\tau}{r_1}{r_2}$
%						and $\exists \rf{\tau}{w_1}{r_1}$
%						then $\rf{\tau}{w_1}{r_2}$ $\v$ $\exists \rf{\tau}{w_2}{r_2}$
%						and $\mo{\tau}{w_1}{w_2}$\newline
%						($\setHB$ between reads forms $\setMO$ between their source
%						writes (if they are different events));
%	\item [\hl{moRW}:] $\forall r_1 \in \reads_\tau$, $w_1 \in \writes_\tau$ \st
%						$\hb{\tau}{r_1}{w_1}$ then $\exists \rf{\tau}{w_2}{r_1}$ and
%						$\mo{\tau}{w_2}{w_1}$\newline
%						($\setHB$ order from a read to a write forms $\setMO$ between the
%						source of the read and the hb-after write);
%	\item [\hl{moWR}:] $\forall w_1 \in \writes_\tau$, $r_1 \in \reads_\tau$ \st
%						$\hb{\tau}{w_1}{r_1}$ then $\rf{\tau}{w_1}{r_1}$ $\v$
%						$\exists \rf{\tau}{w_2}{r_1}$ and $\mo{\tau}{w_1}{w_2}$\newline
%						($\setHB$ order from a write to a read forms $\setMO$ between
%						the write and the source of the read (if the two are
%						different events)).
%\end{itemize}
Further, we introduce an irreflexive relation \sc{\em -from-read} ($\setFR$) to relate  
\sc reads with \sc writes ordered {\em after} it.

\begin{definition}[\sc{\em -from-read} $\setFR$]\newline
	$\forall$ $r^\sc \in \ordreads{\sc}_\tau$, $w^\sc \in \ordwrites{\sc}_\tau$
	if $(r^\sc,w^\sc)$ $\in$ $\setRF^{-1};\setMO$ then $\fr{\tau}{r^\sc}{w^\sc}$.
\end{definition}

A valid \cc execution must also form a total order on all events with memory
order \sc, called \sc{\em -total-order} ($\setTO$) \st
\begin{itemize}[label=to,align=left,leftmargin=*]
	\item [\hl{toHbMo}:] %$\forall e^\sc_1, e^\sc_2 \in \ordevents{\sc}_\tau$, 
			$\to{\tau}{e^\sc_1}{e^\sc_2}$ $\implies$ $\nhb{\tau}{e^\sc_2}{e^\sc_1}$ 
			$\^$ $\nmo{\tau}{e^\sc_2}{e^\sc_1}$\newline
			(\sc-total-order is coherent \wrt $\setHB$ and $\setMO$)
	\item [\hl{toFr}:] $\fr{\tau}{r^\sc}{w^\sc}$ $\implies$ $\to{\tau}{r^\sc}{w^\sc}$
			(\sc-total-order is coherent \wrt $\setFR$)
	\item [\hl{toRf}:] if %$\exists r^\sc \in \ordreads{\sc}_\tau$, $w_1 \in \writes_\tau$ \st
			$\rf{\tau}{w_1}{r^\sc}$ then $\nexists$ $w^\sc \in \ordwrites{\sc}_\tau$
			\st $\to{\tau}{\mo{\tau}{w_1}{w^\sc}}{r^\sc}$.\newline
			(an \sc read must read from its immediate \lmo before write)
\end{itemize}
%

In our technique we use the \lto-rules to build a possibly cyclic
$\setSO$ order between \sc program events and the newly inserted \sc fences.
The cyclicity in the $\setSO$ order invalidates the buggy sequence
under \cc, explained in Section~\ref{sec:so theory}.

\section{title for sc theory} \label{sec:theory}
\section{Methodology} \label{sec:methodology}
\section{Implementation Details} \label{sec:implementation}
\section{Results} \label{sec:results}
\section{Related Work} \label{sec:related}
\section{Conclusion} \label{sec:conclusion}

%\bibliographystyle{unsrtnat}
\bibliography{References.bib}


\end{document}
