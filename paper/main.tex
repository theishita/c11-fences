% This is samplepaper.tex, a sample chapter demonstrating the
% LLNCS macro package for Springer Computer Science proceedings;
% Version 2.20 of 2017/10/04
%
\documentclass[runningheads]{llncs2e/llncs}
%
\input{header.tex}
% Used for displaying a sample figure. If possible, figure files should
% be included in EPS format.
%
% If you use the hyperref package, please uncomment the following line
% to display URLs in blue roman font according to Springer's eBook style:
% \renewcommand\UrlFont{\color{blue}\rmfamily}

\begin{document}
%
\title{Optimal Fence Synthesis for C/C++11}
%
%\titlerunning{Abbreviated paper title}
% If the paper title is too long for the running head, you can set
% an abbreviated paper title here
%
\maketitle              % typeset the header of the contribution
%
\begin{abstract}
The abstract should briefly summarize the contents of the paper in
150--250 words.

\keywords{\cc  \and Fence Synthesis \and Another keyword.}
\end{abstract}
%
%
%
\section{Introduction} \label{sec:intro}
\snj{Could be useful.}
\cite{meshman2014synthesis} show that 'minimum fence insertion problem with multiple types of fence instructions is NP-hard even for straight-line programs' (modeled as the minimum multi-cut problem).

\section{Preliminaries} \label{sec:preliminaries}
Consider a\deleted{n acyclic} multi-threaded \cc
program \added{$P$ $:=$ $\parallel_{i\in \text{\tt TID}} P_i$, 
where $\text{\tt TID}= \{1,\ldots,n\}$ is the set of thread ids}. 
\deleted{The} \added{Each} thread \deleted {of the program}
\added{$P_i$ is a loop-free program, which}
performs a sequence of memory access operations on a set of shared
memory objects and \cc memory fences.  The memory access operations
can be atomic or non-atomic in nature.
%
An instance of a thread operation in an execution is called an {\em
event}.  Events of a thread $t$ are uniquely indexed with an id.
%
\begin{definition}[Event]\newline
An event $i$ of thread $t$ is represented by a tuple $\langle i, t, act, obj,$ $ ord, inst \rangle$ where:
\begin{itemize}[label=inst,align=left,leftmargin=*]
\item [$act$] represents the event action $\in \{ \text{\tt read}, \text{\tt write}, \text{\tt rmw}, \text{\tt fence} \} $,
\item [$obj$] is the set of memory objects accessed,
\item [$ord$] records the \cc memory order associated with the event, and
\item [$inst$] is the corresponding program instruction.
\end{itemize}
\end{definition}
The $act$ {\tt rmw} represents {\em read-modify-write}.
%
Note that the set of memory objects of an rmw event can be non-singleton 
and for a fence event it is an empty set.
%
Let $\events$ denote the set of all program events. Furthermore,
$\writes$, $\reads$ and $\fences$ denote the write, read and fence 
events of the input program.
%
Throughout the text, use of read event as well as write event includes rmw
events unless specified otherwise.
%
\begin{definition}[Trace]\newline
\svs{The definition is cyclic, it uses $\events_\tau$ to define $\tau$. Also, I am not convinced why 
rf and mo together are insufficient and why we need hb additionally? }
\snj{$\events_\tau$ is just a representative symbol, it is not parametric on $\tau$.
The set $\events_\tau \subseteq \events$.}
	A trace or a maximal execution (or simply execution) $\tau$ of the input 
	program $P$ under \cc is a tuple 
	$\langle \events_\tau, \setHB, \setMO, \setRF \rangle$, where
	\begin{itemize}[label=sethb,align=left,leftmargin=*]
		\item [$\events_\tau$] represents the set of events in the trace $\tau$,
		\item [$\setHB$] ({\em Happen-before} relation) is a partial order on
			$\events_\tau$ representing the event interactions and inter-thread
			synchronizations, discussed in Section~\ref{sec:c11},
		\item [$\setMO$] ({\em Modification-order}) is a total order on the
			writes of an object that establishes coherence of $\tau$ 
			\wrt $\setHB$ , and
		\item [$\setRF$] ({\em Reads-from}) is a relation from a write event to
			a read event signifying that the read event takes the value of 
			the write event in $\tau$.
	\end{itemize}
\end{definition}

\noindent
{\bf Memory ordering under \cc}: 
The memory access and fence operations are
associated with ordering modes
that defines the ordering restriction \added{ on them.}
\snj{Technically, the restriction is on the events around the
current event, as stated in the following deleted line:}
\deleted{ placed on atomic and non-atomic access around atomic memory access.}
%
$\moset$ = $\{ \na, \rlx, \rel, \acq, \acqrel, \sc \}$, 
represents the orders relaxed (\rlx), release (\rel), acquire (\acq),
acquire-release (\acqrel) and sequentially consistent (\sc) for
atomic events. A non-atomic event has \na memory order associated with 
it.
%
We use $\ordevents{m}_\tau$ (and accordingly $\ordwrites{m}_\tau$, 
$\ordreads{m}_\tau$ and $\ordfences{m}_\tau$) to represent the $m$
ordered events of an execution sequence $\tau$ (where, $m \in \moset$);
for example $\ordwrites{\rel}_\tau$ represents the write events of 
$\tau$ with ordering restriction \rel. 
\svs{Why not use $o \in \moset$ representing order as a replacement?}
\snj{It makes the definitions very long and difficult to follow, 
eg: let $x \in \events_\tau$
$\^$ $ord(x) = \rel$ vs let $x \in \ordevents{\rel}_\tau$}

%\deleted{
%\noindent
%{\bf \cc fences}: \cc provides atomic thread fences or simply 
%fences to provide additional reordering restrictions on program 
%events. Note that \cc fences are not memory barriers and do not
%provide support for flushing local write values to shared memory.
%%
%A fence can be associated with memory orders $\acqrel$ and $\sc$
%providing varying degrees of reordering restrictions.}

\noindent
{\bf Buggy and fixed executions}: A program $P$ may contain {\em assert} 
instructions as correctness specification. 
\added{$P$ is considered buggy} when a trace of $P$ 
\deleted {that violates an 
assert check (\ie the condition in the assert check computes} \added{has an
assert expression evaluating} to
{\em false}. \deleted{is called a buggy trace.}
\snj{We are more interested in defining a buggy `traces'. 
So we can add in the end that such traces are called `buggy traces'.}


\textcolor{Maroon}{The purpose of this work 
is to synthesize \cc fences at appropriate
program locations to invalidate buggy traces. Particularly, the
event relation in the buggy traces with synthesized fences
render the resulting program behavior invalid under \cc, thus ensuring
that a previously buggy trace would not materialize as a 
\cc program execution.} 
%\comment{Consider moving the above para to the Intro!}
\snj{We can state it in intro as well as here and thus keep reiterating our goal
to make it clear to the reader (provided we have the space.)}

\deleted{We represent the fixed trace corresponding to a
buggy trace $\tau$ by $\inv{\tau}$. 
%
As an intermediate step between $\tau$ and $\inv{\tau}$, we form an 
intermediate version of the trace $\tau$ with candidate fences
some of which are retained as a part of $\inv{\tau}$. We represent
the intermediate version of $\tau$ as $\imm{\tau}$. The details
of the intermediate step are discussed in Section~\ref{sec:methodology}.
%
We also use $\imm{P}$ and $\fx{P}$ to represent the intermediate and
fixed versions of the input program $P$.} \svs{I don't like the above para.}
\snj{The paragraph has been rewritten below:}


In the processes of invalidating a buggy trace $\tau$ of a program $P$,
$\tau$ undergoes two versions of transformations, an intermediate version
(represented as $\imm{\tau}$) and a final `invalidated' version (represented
as $\inv{\tau}$). The transformations have been discussed in 
Sections~\ref{sec:methodology} and \ref{sec:implementation}.
%
Note that once all the buggy traces $\tau$ have been transformed to 
the invalidated version
$\inv{\tau}$ by adding appropriate fences, we then consider the input 
program $P$ {\em fixed} or free of bugs (represented as $\fx{P}$). 

\section{Background: C11 Memory Model} \label{sec:c11}
\input{section/c11_memory_model.tex}

\section{Invalidating buggy sequences with fences} \label{sec:so theory}
\input{section/so_theory.tex}

\section{Methodology} \label{sec:methodology}
% --------- INTRO PART -------------
%\divComment{This para is motivation. Not needed here}
%We have developed a novel solution to the fence-synthesis problem. 
%To the best of our knowledge, 
%there are no push-button fence synthesis techniques for C11 
%programs as of now. Although, there are some such techniques for 
%other memory models, all of those heavily depends on the 
%possible set of reorderings allowed under the memory model. 
%Since the possible set of reorderings under C11 is different 
%than other memory models, these techniques cannot be applied 
%to C11 as is. In face, not all C11 behaviors can be justified 
%using reorderings and interleavings. Hence, automatic fence insertion 
%for C11 requires some amount of additional effort.

% Overview:
%% why only sc fences: if sc fence can't stop none can
%% brief overview of our approach
% Sections:
%% counter example generator and its specifications/requirements
%% inserting candidate fences and computing SO
%%% mention a trivial solution here
%% Fence minimization

In this section we discuss our approach of optimal \mosc fence synthesis 
in \cc programs. 
Our analysis focuses on \mosc fences since these are the strongest \cc 
fences. If a behavior cannot be stopped using \mosc fences, no combination 
of \cc fences can stop this behavior. 
In such cases, our approach proves that the behavior cannot be 
stopped using \cc fences.
Recall that \cc allows memory operations to be annotated with different 
memory orders. For the programs where buggy behavior cannot be stopped 
using fences, it is possible to stop the behavior by annotating some 
memory operations with stronger memory order.
Hence, we do not claim that if \ourtechnique cannot stop a buggy \cc 
trace, it cannot be stopped.
\divComment{Can we give termination guarantee?}

Unlike common fence synthesis approaches \divComment{Need references}, 
\ourtechnique does not try to \emph{add} fences optimally in a program.
Rather, it assumes that the input program has \mosc 
fences at all possible program locations, i.e, before and after every 
program instruction. We call these fences candidate fences. Our technique 
tries to \emph{remove} the unnecessary fences from the set of candidate 
fences based on axiomatic relations of \cc. 
Our approach grantees that the program resulted after removing the 
unnecessary fences, has minimal number of \mosc fences required in that 
program to stop the buggy traces. 

\ourtechnique takes a set of counter examples as input.
We use \cc 's $\setSB$, $\setHB$, $\setMO$, $\setRF$ relations to compute 
$ \setSO $-order assuming the existence of candidate fences.
We prove that the transitive closure of $ \setSO $-order is the same as 
$ \setTO $ in any valid \cc trace. 
Since $ \setTO $ is a total order, any valid \cc trace can not have a 
cycle formed by transitive sequence of ordered pair in $ \setSO $-order. 
We call such cycles $\setSO$-cycle. 
We use $ \setSO $-order and $ \setSO $-cycle to generate an SAT formula. 
Any satisfying assignment of this SAT formula is a possible fence 
placement to stop the buggy trace. We find the minimal satisfying assignment of this SAT formula to find optimal fence placement. 

\begin{algorithm}
	\caption{\ourtechnique}
	\begin{algorithmic}[1]	
		\Procedure{\ourtechnique}{$ P $}
		\State $\phi := \top$
		\State $ \mathcal{CE} := $CounterExampleGenerator($P$)
		\For{$ \tau \in \mathcal{CE} $}\Comment{$\tau$ = $ \langle \events_\tau, \setHB, \setMO, \setRF, \setSB  \rangle $}
			\State $ \events_{\imm{\tau}} := \events_\tau\ \union $ candidate fences
			\State $ \seqb{\imm{\tau}}{}{} := $ computeSB($\setSB, \events_{\imm{\tau}}$) \State $ \so{\tau^{\mathtt{im}}}{}{} := $ computeSO($\events_{\imm{\tau}}, \setHB, \setMO, \setRF, \seqb{\imm{\tau}}{}{}$)
			\State cycles := computeCycles($ \so{\imm{\tau}}{}{} $)
			\If {cycles == $ \emptyset $}
				\State \texttt{Abort} (``This behavior can't be stopped using \cc fences.'')
				\State \Return
			\EndIf
			\State $\phi := \phi\ \^ \formula{\so{\imm{\tau}}{}{}} $ 
%			\State $ \phi := \phi_\tau $
		\EndFor
		\State F:= MinModel($ \phi $)
		\State \Return F
%		
%		\State create\_\z\_file()
%		\State fences\_required = run\_\z\_file()
%		\State insert\_into\_input\_file(fences\_required)
		\EndProcedure
		\label{alg:fence-syn}
	\end{algorithmic}
\end{algorithm}

The \ourtool uses Algorithm~\ref{alg:fence-syn} to insert fences in an 
input program $ P $. The algorithm start by initializing the 
SAT formula $ \phi $ with true.  Line 2 uses a counter-example generator 
to compute the set of counter examples in the program $ P $. 
Lines 4-12 generates SAT formula for each counter-example $ \tau $.
Lines 5 and 6 add candidate fences in the set of events of counter-example 
$ \tau $ and compute $ \lsb $ in the updated trace $ \imm{\tau} $ (i.e., with candidate 
fences). 
Line 7 computes $ \so{\imm{\tau}}{}{} $-order in counter example $ \tau $ with candidate fences.
Line 8 finds all $ \so{\imm{\tau}}{}{} $-cycles. If there are no $ \so{\imm{\tau}}{}{} $-cycles 
present, the behavior can't be stopped used any \cc fences. Hence, we 
terminate the algorithm at this point.
Else, in line 12, we generate a SAT formula for the set of $ \so{\imm{\tau}}{}{} $-cycles and add it to formula $ \phi $
Lastly, in line 13 we find the minimal model of formula $ \phi $, which 
gives us the minimal set of fences to be inserted in order to stop the 
buggy behaviors.
The rest of this section discuss each of these steps in detail.

\subsection{Generating Counter Examples:} 
\ourtechnique requires a set of counter examples $ \mathcal{CE} $. 
Each of these counter-examples $ \tau $ should consist a list program events 
$ \events_\tau $ and $ \setHB, \setMO, \setRF, \setSB $ relations over the 
program events. 
A program event consists of instruction label, action, memory order, variable, and value. 
\divComment{We don't actually need value since we have rf and other relations already.}
Our approach treats a counter example generator that reports all of this information as a black box. 
Any technique that can generate \cc traces and find buggy behaviors can be used as counter-example generator.
%\divComment{Should we also include $ \setSB $ in counter example? Or we can assume it is available in $ \events_\tau $?}
Counter-examples in \texttt{dekker} program are given in 
Figure~\ref{fig:dekker-ce}. We have omitted \lsb, \lhb from the traces for 
simplicity. 

%%%% Move to motivation %%%%
%However, this comes at the cost of making program 
%stricter than required. Therefore, in our solution to retain maximum 
%flexibility, after inserting fences at all places, we must now eliminate 
%the ones which are not required and bring the number of fences down to 
%an optimum or minimum. To do this, we need to find relations between 
%all instructions and understand their semantics.
%
%The reason for choosing \mosc fences is that they are the strongest type 
%of fences, as seen in Fig \ref{fig:mo_strength}. In the program described 
%in Fig \ref{fig:dekker3}, the \moar fences cannot stop the buggy behavior.
%
%Another point to note is, inserting \mosc fences at all possible program 
%locations might not mean that the buggy behaviors are eliminated. 
%Therefore, our tool can only prevent the assertions from being violated 
%in cases where the behavior can be stopped using \mosc fences. Note that
%if a trace cannot be stopped by \mosc fence, then no C11 fence can stop 
%the trace. The tool also serves to minimize the total number of fences to 
%be added to the optimal number, thereby retaining the program behaviors 
%as relaxed as possible.
%%%%%%%%%%%%%%%%%%

\begin{figure}[!htb]
	\begin{center}
		\texttt{(Dekker)} \\ \ \\
		\begin{tabular}{l||l}
			$ \store{y}{1}{} $ & $ \store{x}{1}{} $ \\
			$ \load{a}{x}{} $ & $ \load{c}{y}{} $ \\
			\textbf{if} $ (a=0) $ & \textbf{if} $ (c=0) $ \\
			\quad $ \store{z}{1}{} $ & \quad $ \store{z}{2}{} $ \\
%			\quad $ \load{b}{z}{} $   & \quad$ \load{d}{z}{} $ \\
			\quad assert($ z=1 $) & \quad assert($ z=2 $) \\
		\end{tabular} 
%		\caption{Dekker's Mutual Exclusion program}\label{fig:dekker}
	\end{center}
\end{figure}

\begin{figure}[!h]
	\input{figures/dekker-ce1.tex}
	\input{figures/dekker-ce2.tex}
	\caption{Counter-examples in Dekker}
	\label{fig:dekker-ce}
\end{figure}

% --------------- FIG all pseudo fences -------------
%\begin{figure}[!htb]
%\begin{center}
%\texttt{
%init y := 0, x := 0;\\
%	\begin{tabular}{c l || c l}
%		& F$\mathtt{_{sc}}$ & & F$\mathtt{_{sc}}$\\
%		(1) & W$\mathtt{_{rel}}$y(1) & (5) & W$\mathtt{_{rel}}$x(1)\\
%		& F$\mathtt{_{sc}}$ & & F$\mathtt{_{sc}}$\\
%		(2) & if R$\mathtt{_{rel}x}$ == 0 & (6) & if R$\mathtt{_{rel}y}$ == 0\\
%		& \qquad F$\mathtt{_{sc}}$ & & \qquad F$\mathtt{_{sc}}$\\
%		(3) & \qquad W$\mathtt{_{rel}}$c(1) & (7) & \qquad W$\mathtt{_{rel}}$c(0)\\
%		& \qquad F$\mathtt{_{sc}}$ & & \qquad F$\mathtt{_{sc}}$\\
%		(4) & \qquad assert($\mathtt{R_{rel}c}$ == 1) & (8) & \qquad assert($\mathtt{R_{rel}c}$ == 0)\\
%		& \qquad F$\mathtt{_{sc}}$ & & \qquad F$\mathtt{_{sc}}$\\
%	\end{tabular}
%}

%\divComment{All of the memory orders are rel. Is that correct?}
%	\caption{Dekker's Mutual Exclusion program}\label{fig:dekker3}
%\end{center}
%\end{figure}


\subsection{Computing SO relation}
\noindent\textbf{Inserting Candidate Fences:}
For each counter-example $ \tau $ of buggy program $ P $, we 
construct an intermediate trace $ \imm{\tau} $ by assuming the existance candidate fences in the trace $ \tau $.
Our approach assumes all possible program locations as a valid location 
for candidate fences. 
Hence the $ \imm{\tau} $ will have at least one \mosc fence before and after every program event in $ \tau $.
It is possible to start with a smaller set of program locations for 
candidate fences. 
In this case our approach will return the optimal fence placement wrt the 
set of candidate fences.
Since we are assuming existence of candidate fences at all possible 
program locations, the trace $ \imm{\tau} $ should be invalid \cc trace 
if $ \tau $ can be stopped using \mosc fences.
Hence, if $ \imm{\tau} $ is not invalid \cc trace for some 
the counter-examples $ \tau $ in $ \mathcal{CE} $, we determine that \mosc 
fences are not sufficient to stop the buggy behaviors of the program.
Recall that \mosc fences are the strongest \cc fences. 
If \mosc fences cannot stop a buggy behavior in a program, we can conclude 
that \cc fences are not strong enough to render this program correct.
Previous works \cite{LahavVafeiadis-PLDI17,Batty-POPL12} suggest that \mosc fences are not sufficient to achieve SC memory 
model. 
Hence it is possible that a valid SC program is buggy under \cc with all 
possible \mosc fences inserted. 
In some of these cases strengthening `the memory order annotated for some 
of the program instructions might render the program correct. 
However, strengthening the memory order annotation is out of the scope of this work. 

%\begin{figure}[!htb]
%	\begin{center}
%%		\texttt{(Dekker)} \\ \ \\
%		\begin{tabular}{l||l}
%			$ \fenceev{11} $ & $ \fenceev{21} $ \\
%			$ \store{y}{1}{} $ & $ \store{x}{1}{} $ \\
%			$ \fenceev{12} $ & $ \fenceev{22} $ \\
%			$ \load{a}{x}{} $ & $ \load{c}{y}{} $ \\
%			$ \fenceev{13} $ & $ \fenceev{23} $ \\
%			\textbf{if} $ (a=0) $ & \textbf{if} $ (c=0) $ \\
%			\quad $ \fenceev{14} $ & \quad $ \fenceev{24} $ \\
%			\quad $ \store{z}{1}{} $ & \quad $ \store{z}{2}{} $ \\
%			%			\quad $ \load{b}{z}{} $   & \quad$ \load{d}{z}{} $ \\
%			\quad $ \fenceev{15} $ & \quad $ \fenceev{25} $ \\
%			\quad assert($ z=1 $) & \quad assert($ z=2 $) \\
%			\quad $ \fenceev{16} $ & \quad $ \fenceev{26} $ \\
%		\end{tabular} 
%		\caption{Dekker with Candidate Fence}\label{fig:dekker-im}
%	\end{center}
%\end{figure}

\noindent\textbf{Computing \lsb in Intermediate Traces:} 
In line 5 of the Algorithm~\ref{alg:fence-syn}, we compute set of program 
events in trace $ \imm{\tau} $. Naturally, this set will consist of 
candidate fences along with all the program events in trace $ \tau $.
The set $ \setSB $ represents the pair of program events related by 
relation $ \lsb $ in the trace $ \tau $. We compute the set 
$ \seqb{\imm{\tau}}{}{} $ for the events of the trace $ \imm{\tau} $. 
The computeSB function takes set of \lsb event pairs in $ \tau $ 
($ \setSB $), the set of events in trace $ \imm{\tau} $ 
($ \events_{\imm{\tau}} $) as input and computes the set of \lsb in 
$ \imm{\tau} $ based on instruction labels, their position in the program $ P $. \divComment{Do we need event action to compute the \lsb?}

\noindent\textbf{Computing \lso in Intermediate Traces:} 
%In \textsection\ref{sec:so theory}, we explain the rules to compute \lso relation given \lhb, \lmo, \lrf and \lsb relations. 
The line 7 of Algorithm~\ref{alg:fence-syn} uses the rules defined in 
\textsection\ref{sec:so theory} to compute the \lso relation in the trace 
$ \imm{\tau} $ using the relations \lhb, \lmo, \lrf and \lsb. 
We use relations $ \setHB, \setMO, \setRF $ and $ \seqb{\imm{\tau}}{}{} $ to compute $ \so{\imm{\tau}}{}{} $.
Note that we have used the \lhb, \lmo and \lrf for trace $ \tau $ and \lsb 
for trace $ \imm{\tau} $. 
Since \lmo is a relation between write events, fences will not be a part 
of it. Although, these fences may induce \lmo between some write event 
pairs in an actual trace. 
\divComment{How do we know these induces \lmo will not effect \lso?}
The same argument hold for \lrf relation as well.
\divComment{Need an argument for \lhb too.}
%We have already computed $ \seqb{\imm{\tau}}{}{} $
 
\snj{Why hb, mo computed for $\tau$ and not $\imm{\tau}$}\newline
Note that, the intermediate trace $\imm{\tau}$ may be invalid under 
\cc due to a possible cycle in the $\so{\imm{\tau}}{}{}$ relation. 
As a result, the coherence rules of \cc do not apply to $\imm{\tau}$. 
In the absence of \cc coherence rules the $\hb{\imm{\tau}}{}{}$ and 
the $\mo{\imm{\tau}}{}{}$ relations cannot be formed on the events of 
$\imm{\tau}$.
%
Secondly, attempting to form $\hb{\imm{\tau}}{}{}$ and 
$\mo{\imm{\tau}}{}{}$ relations would modify the $\setHB$ and $\setMO$
effectively changing the buggy trace $\tau$.

We, thus, only construct the relation $\so{\imm{\tau}}{}{}$ and 
$\seqb{\imm{\tau}}{}{}$ (needed for $\so{\imm{\tau}}{}{}$) 
between the candidate fences inserted in $\imm{\tau}$ and the events
of $\tau$. The two relations are sufficient as the goal is to
find $\so{\imm{\tau}}{}{}$ cycle and invalidate the trace under \cc.


\subsection{Reducing Fence Synthesis to SAT problem}
Recall that transitive closure of $ \so{\imm{\tau}}{}{} $ is same as 
$ \to{\imm{\tau}}{}{} $ in a valid \cc trace. Since  
$ \to{\imm{\tau}}{}{} $ is a total order, a valid \cc trace should not 
have a cycle in $ \so{\imm{\tau}}{}{} $. Conversely, if a trace has 
$ \so{\imm{\tau}}{}{} $-cycle, it cannot be a valid \cc trace. 
In order to make a trace $ \tau $ invalid under \cc, we force a \lso-cycle by inserting appropriate fences. 
Since $ \imm{\tau} $ assumes \mosc fences at all possible program locations,
any such cycle must exists in $ \so{\imm{\tau}}{}{} $. 
Hence, our problem is reduced to finding appropriate cycle in 
$\so{\imm{\tau}}{}{} $ and introducing these fences in the program to 
invalidate the trace $ \tau $.
If we introduced enough fences to cause at least \lso-cycle in 
every counter example, we stop all the buggy trace. 
It is possible that a counter example $ \tau $ does not have any 
$\so{\imm{\tau}}{}{}$-cycles. Since $ \imm{\tau} $ has \mosc fences at all 
possible program locations, we cannot add fences in such a trace to 
make it invalid \cc execution. 
Line 8 in Algorithm~\ref{alg:fence-syn} computes set $\so{\imm{\tau}}{}{}$-cycles. 
The problem of finding cycles in a graph is well-studied area. Hence, we 
choose to skip the details of this step.
If there are no $\so{\imm{\tau}}{}{}$-cycle in some $ \tau $, we conclude 
that it is not possible to stop this behavior using \cc fences in lines 
9-11.


Let $ \cycles{\imm{\tau}} $ be the set of \lso-cycles in a trace $ \imm{\tau} $, 
where a cycle $ c \in \cycles{\imm{\tau}} $ is ordered sequence of 
$ \ordevents{\sc} $. We abuse the notation $\ordevents{\sc}_c$ to 
represent the set of events in a cycle $ c $. 
All the read and write events in $\ordevents{\sc}_c$ are already in program $ P $. 
Hence, to introduce a cycle $ c $ in the execution, we add the fences in $\ordevents{\sc}_c$, i.e, 
$\ordfences{\sc}_c$, in the program $ P $.
In other words, a cycle $ c $ can be introduced in a program if we insert 
$ (\bigwedge\limits_{f \in \ordfences{\sc}_c} f)$ fences in the program $ P $, 
where the truth assignment to a fence corresponds to inserting that fence 
in the program.
Recall that we need to stop at least one cycle from the set 
$ \cycles{\imm{\tau}} $ in order to make a trace $ \tau $ invalid.
Hence we need to insert $ (\bigvee\limits_{c \in \cycles{\imm{\tau}}} (\bigwedge\limits_{\fn{} \in \ordevents{\sc}_c \intersection 
\ordfences{\sc}} \fn{})) $ to stop a trace $ \tau $. 
We use $\formula{\so{\imm{\tau}}{}{}}$ to represent the boolean formula 
corresponding to the $ \so{\imm{\tau}}{}{} $-cycles.
Clearly, any satisfying assignment to $ \formula{\so{\imm{\tau}}{}{}} $ 
will give list of fences required to stop the buggy execution $ \tau $.
%
Furthermore, we construct the formula for all trace 
$ \tau \in \mathcal{CE} $ by conjuncting them in 
line 12 of Algorithm~\ref{alg:fence-syn}, i.e., 
at the end of the for loop in lines 4-12, the formula 
$ \phi \definedas (\bigwedge\limits_{\tau \in \mathcal{CE}} 
(\bigvee\limits_{c \in \cycles{\imm{\tau}}} 
(\bigwedge\limits_{\fn{} \in \ordevents{\sc}_c \intersection \ordfences{\sc}} \fn{}))) $.
Any satisfying assignment of the formula $\phi$ will list the fences that 
are enough to stop the buggy behaviors in $ \mathcal{CE} $.

%\begin{lemma}
%	Introducing fences in any $\so{\imm{\tau}}{}{}$-cycle will stop the counter example $ \tau $ in actual execution of the program.
%\end{lemma}

\begin{lemma}
	If there are no cycles in $\so{\imm{\tau}}{}{}$ of a buggy execution 
	$\tau$, the counter-example $ \tau $ cannot be stopped using any \cc 
	fences.
\end{lemma}

\begin{theorem}
	Any satisfying assignment to $ \formula{\so{\imm{\tau}}{}{}} $ 
	will give list of fences required to stop the buggy execution $ \tau $
\end{theorem}


\subsection{Finding Optimal Placement of the Fences}
One possible satisfying assignment to formula $ \phi $ is assigning the 
value true to all fences. Clearly such a solution is very expensive.
The number of truth assignments in the solution of formula $ \phi $ is 
equal to the number of fences inserted in the program.
%A satisfying assignment of the formula $ \phi $ with lesser number 
%of fences will give us a more optimal fence placement. 
Hence, a satisfying assignment of the formula $ \phi $ with the least 
number of fences will give us the optimal fence placement. 
Therefore, we reduce the problem of finding optimal fence placement to the 
minimal model of a SAT formula. 

The problem of minimal model computation has been studied in 
\divComment{need references}. \divComment{Need complexity argument for min model}.

In some cases, a fence at certain program location maybe more expensive 
than at other program location. For example, a fence inside a loop is may 
execute several times. Hence, such a fence may be more expensive than a 
fence placed outside the loop body.
%
\ourtechnique can handle such constraints by using weighted minimal model 
problem \divComment{need references}, where weight of each fence 
corresponds how expensive the fence is. 
The Algorithm~\ref{alg:fence-syn} can be modified to take weight 
function as input. 
%
The current implementation of \ourtechnique uses repeated calls to Z3 SAT 
solver to find the minimal model. \divComment{How are we solving it?}


\begin{theorem}
	For any input program $ P $, minimal model of $ \phi $ gives optimal
	number of fences required to stop all the buggy behaviors in 
	$ \mathcal{CE} $.
\end{theorem}



\section{Implementation Details} \label{sec:implementation}
\input{section/implementation.tex}

\section{Results} \label{sec:results}
\begin{table}
		\resizebox{\columnwidth}{!}{%
\begin{tabular}{l|r|r|r|r|r|r|r|l}
\multicolumn{1}{c|}{}                       & \multicolumn{1}{c|}{}                                                                             & \multicolumn{1}{c|}{}                                                                            & \multicolumn{1}{c|}{}                                                                                                & \multicolumn{4}{c|}{\textit{Times (in seconds)}}                                                                        & \multicolumn{1}{c}{}                                    \\ \cline{5-8}
\multicolumn{1}{c|}{\multirow{-2}{*}{Name}} & \multicolumn{1}{c|}{\multirow{-2}{*}{\begin{tabular}[c]{@{}c@{}}Buggy\\ executions\end{tabular}}} & \multicolumn{1}{c|}{\multirow{-2}{*}{\begin{tabular}[c]{@{}c@{}}Fences\\ inserted\end{tabular}}} & \multicolumn{1}{c|}{\multirow{-2}{*}{\begin{tabular}[c]{@{}c@{}}Avg no. of\\ instr/trace\end{tabular}}} & \multicolumn{1}{c|}{\cite{cds}} & \multicolumn{1}{c|}{Z3} & \multicolumn{1}{c|}{Tool only} & \multicolumn{1}{c|}{Total} & \multicolumn{1}{c}{\multirow{-2}{*}{\textit{(source)}}} \\ \hline
dekker\_no\_fence                           & 2                                                                                                 & 2                                                                                                & 19                                                                                                                   & 0.30                            & 0.02                    & {\color[HTML]{00009B} 0.02}    & 0.34                       & \cite{cds}                                              \\
fib\_mod\_false-unreach-call                & 12                                                                                                & 3                                                                                                & 32                                                                                                                   & 4.44                            & 0.36                    & {\color[HTML]{00009B} 1.93}    & 6.74                       & \cite{cds}                                              \\
mot\_eg\_modified                           & 292                                                                                               & 4                                                                                                & 40                                                                                                                   & 0.73                            & 0.12                    & {\color[HTML]{00009B} 3.99}    & 4.83                       & \cite{cds}                                              \\
mot\_eg\_v2\_2                              & 1835                                                                                              & 2                                                                                                & 32                                                                                                                   & 2.56                            & 0.03                    & {\color[HTML]{00009B} 12.18}   & 14.77                      & \cite{cds}                                              \\
mot\_eg\_v3                                 & 19                                                                                                & 2                                                                                                & 26                                                                                                                   & 0.26                            & 0.02                    & {\color[HTML]{00009B} 0.10}    & 0.37                       & \cite{cds}                                              \\
mot\_eg\_v3\_modified                       & 29                                                                                                & 5                                                                                                & 27                                                                                                                   & 0.25                            & 0.02                    & {\color[HTML]{00009B} 0.14}    & 0.41                       & \cite{cds}                                              \\
peterson                                    & 24                                                                                                & 2                                                                                                & 24                                                                                                                   & 0.28                            & 0.03                    & {\color[HTML]{00009B} 0.67}    & 0.98                       & \cite{cds}                                              \\
read\_write\_lock\_2                        & 524                                                                                               & 4                                                                                                & 41                                                                                                                   & 1.42                            & 6.60                    & {\color[HTML]{00009B} 84.69}   & 92.70                      & \cite{cds}                                              \\
read\_write\_lock\_unreach\_11              & 1                                                                                                 & 2                                                                                                & 27                                                                                                                   & 0.24                            & 0.02                    & {\color[HTML]{00009B} 0.02}    & 0.28                       & \cite{cds}                                              \\
read\_write\_lock\_unreach\_12              & 58                                                                                                & 2                                                                                                & 34                                                                                                                   & 0.43                            & 0.02                    & {\color[HTML]{00009B} 0.55}    & 0.99                       & \cite{cds}                                              \\
read\_write\_lock\_unreach\_13              & 9229                                                                                              & 2                                                                                                & 44                                                                                                                   & 55.51                           & 0.09                    & {\color[HTML]{00009B} 153.14}  & 208.74                     & \cite{cds}                                              \\
basic                                       & 1                                                                                                 & 2                                                                                                & 21                                                                                                                   & 0.33                            & 0.01                    & {\color[HTML]{00009B} 0.01}    & 0.35                       &                                                         \\
mixed\_eg                                   & 4                                                                                                 & 4                                                                                                & 27                                                                                                                   & 0.33                            & 0.02                    & {\color[HTML]{00009B} 0.06}    & 0.41                       &                                                         \\
dekker\_rlx                                 & 2                                                                                                 & 2                                                                                                & 17                                                                                                                   & 0.32                            & 0.01                    & {\color[HTML]{00009B} 0.35}    & 0.01                       &                                                         \cite{rcmc-POPL18}														\\
pgsql0                                      & 7                                                                                                 & 2                                                                                                & 23                                                                                                                   & 0.27                            & 107.13                  & {\color[HTML]{00009B} 603.75}  & 496.35                     & \cite{genmc-PLDI19}                                                   \\
publish-sc0                                 & 41                                                                                                & 3                                                                                                & 55                                                                                                                   & 0.36                            & 0.01                    & {\color[HTML]{00009B} 4.60}    & 4.24                       & \cite{genmc-PLDI19}                                                   \\
publish-sc1                                 & 41                                                                                                & 3                                                                                                & 55                                                                                                                   & 0.30                            & 0.01                    & {\color[HTML]{00009B} 4.56}    & 4.25                       & \cite{genmc-PLDI19}                                                   \\
SB+assert0                                  & 1                                                                                                 & 2                                                                                                & 15                                                                                                                   & 0.32                            & 0.01                    & {\color[HTML]{00009B} 0.34}    & 0.01                       & \cite{genmc-PLDI19}                                                   \\
SB+assert1                                  & 1                                                                                                 & 2                                                                                                & 15                                                                                                                   & 0.32                            & 0.01                    & {\color[HTML]{00009B} 0.34}    & 0.01                       & \cite{genmc-PLDI19}                                                   \\
thread01                                    & 2                                                                                                 & 2                                                                                                & 13                                                                                                                   & 0.32                            & 0.02                    & {\color[HTML]{00009B} 0.35}    & 0.01                       & \cite{watts}                                                   \\
2+2W0053                                    & 1                                                                                                 & 2                                                                                                & 22                                                                                                                   & 0.33                            & 0.02                    & {\color[HTML]{00009B} 0.36}    & 0.02                       & \cite{tracer2018}                                                  \\
2+2W                                        & 1                                                                                                 & 2                                                                                                & 18                                                                                                                   & 0.32                            & 0.01                    & {\color[HTML]{00009B} 0.35}    & 0.01                       & \cite{tracer2018}                                                  \\
3.2W+lwsync+lwsync+po                       & 1                                                                                                 & 3                                                                                                & 28                                                                                                                   & 0.33                            & 0.02                    & {\color[HTML]{00009B} 0.37}    & 0.02                       & \cite{tracer2018}                                                  \\
DETOUR0928                                  & 1                                                                                                 & 2                                                                                                & 29                                                                                                                   & 0.33                            & 0.02                    & {\color[HTML]{00009B} 0.37}    & 0.02                       & \cite{tracer2018}                                                  \\
m2                                          & 9                                                                                                 & 2                                                                                                & 28                                                                                                                   & 0.33                            & 0.01                    & {\color[HTML]{00009B} 0.47}    & 0.12                       & \cite{tracer2018}                                                  \\
mc                                          & 3                                                                                                 & 2                                                                                                & 39                                                                                                                   & 0.34                            & 0.01                    & {\color[HTML]{00009B} 0.46}    & 0.10                       & \cite{tracer2018}                                                  \\
MOREDETOUR0398                              & 4                                                                                                 & 2                                                                                                & 43                                                                                                                   & 0.45                            & 0.02                    & {\color[HTML]{00009B} 0.64}    & 0.18                       & \cite{tracer2018}                                                  \\
MOREDETOUR0406                              & 3                                                                                                 & 2                                                                                                & 47                                                                                                                   & 0.55                            & 0.02                    & {\color[HTML]{00009B} 0.74}    & 0.17                       & \cite{tracer2018}                                                  \\
MOREDETOUR0685                              & 5                                                                                                 & 2                                                                                                & 48                                                                                                                   & 0.93                            & 0.02                    & {\color[HTML]{00009B} 1.28}    & 0.33                       & \cite{tracer2018}                                                  \\
MOREDETOUR0687                              & 1                                                                                                 & 2                                                                                                & 36                                                                                                                   & 0.34                            & 0.02                    & {\color[HTML]{00009B} 0.40}    & 0.04                       & \cite{tracer2018}                                                  \\
MOREDETOUR0874                              & 2                                                                                                 & 2                                                                                                & 38                                                                                                                   & 0.36                            & 0.03                    & {\color[HTML]{00009B} 0.48}    & 0.10                       & \cite{tracer2018}                                   
\end{tabular}%
}
	\caption{Bare bone results}
	\label{tab:results1}
\end{table}

In order to analyze and evaluate the performance of the algorithm 
proposed in this paper, we created a tool called ----. 77 benchmarks were
executed on it, using a computer with \ishComment{specs}.
Most of these benchmarks have been acquired from verification tool
repositories, namely CDSChecker \cite{cds}, Tracer \cite{tracer2018},
GenMC \cite{genmc-PLDI19}, RCMC \cite{rcmc-POPL18}, Watts \cite{watts}.

All benchmarks can be segregated into two base classes. The first set consists of 
programs whose buggy executions can be stopped by inserting fences. Table \ref{tab:results1}
summarizes these results. The second category of benchmarks contains programs which
either do not provide obtainable results using a model checking tool, or
whose buggy behaviours cannot be stopped simply by using fences. \textit{\{The reason
for this may be because those programs are semantically created to prove
a point instead of being stopped using fences.\}} Benchmarks range from
having a few buggy executions to having thousands of buggy executions. They
also vary in the number of threads and atomic instructions in each of
their traces.

For each concurrent program, we provide the following information as criteria for evaluation:
the number of buggy executions found by the model checker, the number of fences
inserted in the final program, and the average number of atomic instructions
in each trace. The tool is assessed by being timed according to different steps
of the program, including time taken by the model checker, time taken by the
SAT solver and the remaining computation time taken by the tool itself. Programs
were run in two modes- the regular mode with all executions
being taken into account and the \texttt{-t} flag mode, where only the first
trace was taken into consideration for each iteration of the input program.

From the experimental result obtained, it can be noted that a major factor directly affecting 
the tool time is the number of buggy executions. This number affects the tool 
time since each execution is computed individually. Naturally, more executions translates to
greater computation time. This is the case, for example, with benchamarks
\texttt{read\_write\_lock\_unreach\_13} and \texttt{mot\_eg\_v2\_2}, which had thousands
of buggy executions. Computing these were time taking tasks for 
both CDSChecker and the tool itself.

\begin{figure}
	a. $O(n^3)$, \\
	\textit{where n is the number of vertices}\\
	
	
	b. $O((n+e)(c+1))$, \\
	\textit{for n nodes, e edges and c elementary circuits}
	\caption{a. complexity of computing transitive \setHB relations\\
	b. complexity of Johnson's algorithm to compute all elementary cycles in a graph}
\end{figure}

Through experiments, it was also observed that the two most expensive 
operations in the entire process were computing transitive relations for \setHB 
relations, and computing all possible elementary cycles in the graph of \setTO
relations. The metric, ``number of instructions in each trace'' adds
considerably to the former factor. A large number of instructions in each
trace produces a greater number of \setHB relations, thereby resulting in
greater computation time for transitive relations. This is evidenced by the benchmarks
\texttt{publish-sc0} and \texttt{publish-sc1} where the number of instructions in each
execution trace is 55 - a greater than average amount for our benchmarks.
Consequently, the tool takes more time solving them.

This metric does not necessarily apply to \setTO relations.
A greater number of instructions may or may not produce a sizeable \setTO graph.
This depends upon the semantics of the input program. However, chances are,
an intricate graph with many \setTO relations will result in an enormous number of cycles, 
sometimes having exponential numbers, resulting in a much greater
computation time. This is evidenced by looking at the Z3 computation 
time column, since the SAT Solver is only concerned with reducing the cycles
to a minimal required number of fences. For instance, the benchmark \texttt{pgsql0}, though having 
a very small number of buggy executions, has an execution time of a few minutes. Much
of this time in spent on SAT Solving, as can be witnessed. This result reveals 
a very large number of \setTO relations and hence cycles.

\begin{table}
\begin{center}
	% Please add the following required packages to your document preamble:
% \usepackage{multirow}
\begin{tabular}{l|r|r|r|r}
\multicolumn{1}{c|}{Name}                           & \multicolumn{1}{c|}{{\color[HTML]{656565} \begin{tabular}[c]{@{}c@{}}Fences\\ inserted\\ (before)\end{tabular}}} & \multicolumn{1}{c|}{\begin{tabular}[c]{@{}c@{}}Fences\\ inserted\\ (after)\end{tabular}} & \multicolumn{1}{c|}{{\color[HTML]{656565} \begin{tabular}[c]{@{}c@{}}Total\\ time\\ (before)\end{tabular}}} & \multicolumn{1}{c}{\begin{tabular}[c]{@{}c@{}}Total\\ time\\ (after)\end{tabular}} \\ \hline
publish-sc0                                         & {\color[HTML]{656565} 3}                                                                                         & 3                                                                                        & {\color[HTML]{656565} 4.60}                                                                                 & 2.24                                                                               \\
{\color[HTML]{000000} publish-sc1}                  & {\color[HTML]{656565} 3}                                                                                         & {\color[HTML]{000000} 3}                                                                 & {\color[HTML]{656565} 4.56}                                                                                 & {\color[HTML]{000000} 2.56}                                                        \\
mot\_eg\_modified                                   & {\color[HTML]{656565} 4}                                                                                         & 5                                                                                        & {\color[HTML]{656565} 4.83}                                                                                 & 2.47                                                                               \\
{\color[HTML]{000000} fib\_mod\_false-unreach-call} & {\color[HTML]{656565} 3}                                                                                         & {\color[HTML]{000000} 3}                                                                 & {\color[HTML]{656565} 6.74}                                                                                 & {\color[HTML]{000000} 228.93}                                                      \\
peterson                                            & {\color[HTML]{656565} 2}                                                                                         & 3                                                                                        & {\color[HTML]{656565} 0.98}                                                                                 & 0.75                                                                               \\
{\color[HTML]{000000} pgsql0}                       & {\color[HTML]{656565} 2}                                                                                         & {\color[HTML]{000000} 2}                                                                 & {\color[HTML]{656565} 603.75}                                                                               & {\color[HTML]{000000} 0.71}                                                        \\
mot\_eg\_v2\_2                                      & {\color[HTML]{656565} 2}                                                                                         & 2                                                                                        & {\color[HTML]{656565} 14.77}                                                                                & 16.03                                                                              \\
{\color[HTML]{000000} read\_write\_lock\_2}         & {\color[HTML]{656565} 4}                                                                                         & {\color[HTML]{000000} 4}                                                                 & {\color[HTML]{656565} 92.7}                                                                                 & {\color[HTML]{000000} 4.53}                                                        \\
read\_write\_lock\_unreach\_13                      & {\color[HTML]{656565} 2}                                                                                         & 2                                                                                        & {\color[HTML]{656565} 208.74}                                                                               & 200.35                                                                            
\end{tabular}
	\caption{Optimized results}
	\label{tab:results2}
\end{center}
\end{table}

To combat this problem of number of cycles or number of buggy executions going out of bounds,
the optimization discussed in \ref{sec:optis} was applied.
Table \ref{tab:results2} contains a few of these results. Results state that
in most cases, the number of fences remains the same as they were with the 
full optimization, but the time is considerably reduced.

The tool performance cannot be compared with any previous approaches since there have been no
approaches for fence synthesis for the C11 model.

\section{Related Work} \label{sec:related}
The literature in the  fence synthesis is dominated by 
techniques targeting the x86-TSO and sparc PSO memory models
such as \cite{abdulla2012counter,alglave2010fences,alglave2014don,linden2011verification,abdulla2012automatic,abdulla2015best,bender2015declarative} 
for x86-TSO,
\cite{abdulla2015precise,linden2013verification} 
for sparc-PSO and \cite{liu2012dynamic,meshman2014synthesis,abdulla2013memorax,joshi2015property,kuperstein2012automatic}
for both TSO and PSO.
%
The technique \cite{bender2015declarative} additionally
provides fence synthesis for ARMv7 memory model and 
\cite{kuperstein2012automatic} additionally provides
fence synthesis for RMO memory model.
%
Apart from the techniques for TSO and PSO, a few 
fence synthesis techniques have been proposed for Power
memory model 
\cite{alglave2010fences,abdulla2015precise,fang2003automatic}, 
where \cite{fang2003automatic} also 
provides support for IA-32 memory model, and
\cite{joshi2015property} provides a technique applicable to
any memory model whose behaviors can be justified using
interleaving with reordering.
%
\ourtechnique is the first fence synthesis technique for 
\cc memory model.

Some of the earlier works  
\cite{alglave2010fences,abdulla2015precise,fang2003automatic} 
perform fence synthesis to reduce weak memory behaviors of an 
input program to those permitted under Sequential Consistency (SC)
or its variant \cite{abdulla2015best}. Such reduction would
make the input compatible with the existing techniques 
verification for SC memory model.
%
Most fence synthesis techniques (including \ourtechnique)
\cite{meshman2014synthesis}\cite{abdulla2012counter}\cite{abdulla2013memorax}\cite{joshi2015property}\cite{abdulla2015precise}\cite{linden2011verification}\cite{kuperstein2012automatic}\cite{abdulla2012automatic}\cite{linden2013verification}, however, 
use safety property specifications (usually provided as assert
statements in the input program) and attempt to remove 
program traces that violate a safety property.

Some of the techniques discussed above
\cite{meshman2014synthesis}\cite{taheri2019polynomial}\cite{abdulla2012counter}\cite{abdulla2013memorax}\cite{joshi2015property}\cite{abdulla2015precise}\cite{kuperstein2012automatic}\cite{bender2015declarative} perform optimal fence synthesis, 
same as \ourtechnique, for their respective memory models.
%
The techniques \cite{linden2011verification}\cite{linden2013verification} 
extend their solution to cyclic programs.
%
The technique \cite{joshi2015property} proposes a bounded 
fence synthesis technique, where they bound the number of 
reordering to a constant $k$.

\section{Conclusion} \label{sec:conclusion}

\bibliographystyle{llncs2e/splncs04}
\bibliography{References}

\end{document}
