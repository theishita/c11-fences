The \cc memory model introduces an irreflexive and acyclic relation 
over the events of a trace $\tau$ called the {\em happens-before} 
relation ($\setHB$),
 \st $\setHB \subseteq \events_\tau {\times} \events_\tau$.
A trace is {\em coherent} (or valid) under \cc if 
the $\setHB$ relation does not violate any {\em coherence rule}
\cite{batty2011mathematizing}\cite{C11}.
In this paper we discuss only subsets of $\setHB$ and 
coherence rules that are relevant to this work.
%
%
Let $\setRF$ represent a {\em reads-from} relation relating a 
write event to a read event, that reads from it in trace $\tau$.
The events of a sequence $\tau$ are related by the $\setHB$ as 
follows;
\begin{itemize}
	\item {\em Intra-thread-hb}: A pair of events in $P_i$ are related by
        their program order are also related by a 
        {\em sequenced-before} relation ($\setSB$).
	
	\item {\em Inter-thread-hb}: When a strict write $e_w$ (memory order
		\rel or stricter) and a strict read $e_r$ (memory order \acq or
		stricter) from different threads are related as 
		$\rf{\tau}{e_w}{e_r}$, they are also 
		related by an {\em inter-thread-hb} relation ($\setITHB$).
		%
		This forms a synchronization between the threads of $e_w$
		and $e_r$. 
		%
		As a consequence, every event $e,e'$ \st $\seqb{\tau}{e}{e_w}$ and 
		$\seqb{\tau}{e_r}{e'}$ are also related as $\ithb{\tau}{e}{e'}$.
		\newline
		%
		The $\setITHB$ relation can also be formed between \cc fences 
		\cite{batty2011mathematizing}\cite{C11}, we skip the details 
		for brevity.
\end{itemize}

Accordingly, two events $e_1,e_2$ in an execution sequence $\tau$ are happens-before 
related \ie
$\hb{\tau}{e_1}{e_2}$ if $\seqb{\tau}{e_1}{e_2}$ $\v$ $\ithb{\tau}{e_1}{e_2}$.
%
Further, in a sequence $\tau$, all write events of an object $x$ are related by a 
total order called {\em modification-order} ($\setMO$).
%
The $\setMO$ order is constructed in compliance to a set of coherence
\lmo-rules defined by \cc \cite{C11}
based on $\setHB$ and $\setRF$. 
%
In other words, {\it in a valid \cc trace the union of $\setHB$, $\setRF$ and 
$\setMO$ must be acyclic}.
%The set of $\setMO$ rules are,
%\begin{itemize}[label=moWW,align=left,leftmargin=*]
%	\item [\hl{moWW}:] $\forall w_1, w_2 \in \writes_\tau$ if $\hb{\tau}{w_1}{w_2}$ then
%						$\mo{\tau}{w_1}{w_2}$\newline
%						($\setHB$ ordered writes are ordered by $\setMO$);
%	\item [\hl{moRR}:] $\forall r_1, r_2 \in \reads_\tau$ \st $\hb{\tau}{r_1}{r_2}$
%						and $\exists \rf{\tau}{w_1}{r_1}$
%						then $\rf{\tau}{w_1}{r_2}$ $\v$ $\exists \rf{\tau}{w_2}{r_2}$
%						and $\mo{\tau}{w_1}{w_2}$\newline
%						($\setHB$ between reads forms $\setMO$ between their source
%						writes (if they are different events));
%	\item [\hl{moRW}:] $\forall r_1 \in \reads_\tau$, $w_1 \in \writes_\tau$ \st
%						$\hb{\tau}{r_1}{w_1}$ then $\exists \rf{\tau}{w_2}{r_1}$ and
%						$\mo{\tau}{w_2}{w_1}$\newline
%						($\setHB$ order from a read to a write forms $\setMO$ between the
%						source of the read and the hb-after write);
%	\item [\hl{moWR}:] $\forall w_1 \in \writes_\tau$, $r_1 \in \reads_\tau$ \st
%						$\hb{\tau}{w_1}{r_1}$ then $\rf{\tau}{w_1}{r_1}$ $\v$
%						$\exists \rf{\tau}{w_2}{r_1}$ and $\mo{\tau}{w_1}{w_2}$\newline
%						($\setHB$ order from a write to a read forms $\setMO$ between
%						the write and the source of the read (if the two are
%						different events)).
%\end{itemize}
%
We introduce an irreflexive relation {\em from-reads} ($\setFR$) to relate  
reads with writes ordered {\em after} it.

\begin{definition}[{\em from-reads} $\setFR$]\newline
	$\forall$ $r \in \reads_\tau$, $w \in \writes_\tau$
	if $(r,w)$ $\in$ $\setRF^{-1};\setMO$ then $\fr{\tau}{r}{w}$.
\end{definition}
% 
{\it A valid \cc execution must also form a total order on all events with memory
order \sc}, called \sc{\em -total-order} ($\setTO$) \cite{LahavVafeiadis-PLDI17} \st
$\forall e^\sc_1, e^\sc_2 \in \ordevents{\sc}$,
\begin{itemize}[label=to,align=left,leftmargin=*]
	\item [\hl{toHb}:]
			$\to{\tau}{e^\sc_1}{e^\sc_2}$ $\implies$ $\nhb{\tau}{e^\sc_2}{e^\sc_1}$ 
			(\sc-total-order is coherent \wrt $\setHB$)
	\item [\hl{toMo}:]
			$\to{\tau}{e^\sc_1}{e^\sc_2}$ $\implies$ $\nmo{\tau}{e^\sc_2}{e^\sc_1}$
			(\sc-total-order is coherent \wrt $\setMO$)
%	\item [\hl{toFr}:] 
%			$\to{\tau}{e^\sc_1}{e^\sc_2}$ $\implies$ $\nfrsc{\tau}{e^\sc_2}{e^\sc_1}$
%			(\sc-total-order is coherent \wrt $\setFR$)
\item $\onsc{\nrightarrow^{{\bf\textcolor{RoyalPurple}{fr}}}}$
	\item [\hl{toRf}:] if 
			$\rf{\tau}{e_w}{e^\sc_2}$ then $\nexists$ $e^\sc_1 \in \ordwrites{\sc}_\tau$
			\st $\to{\tau}{\mo{\tau}{e_w}{e^\sc_1}}{e^\sc_2}$.\newline
			(an \sc read must read from its immediate \lmo before write)
\end{itemize}
%

In our technique we attempt to break the acyclicity requirement over
$\setHB$ and $\lto$-rules by strategically placing \cc fences in 
the input program, as discussed in Section~\ref{sec:so theory}.

\snj{add some intro to fences and fence mo orders.}
