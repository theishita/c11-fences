The C11 standard \cite{C11} associates each atomic event with 
a {\em memory order}. The memory orders specify how atomic and
non-atomic events can be ordered around an atomic event placing
restrictions on the feasible program outcomes.
%
Let $\moset = \{\na,\morlx,\morel,\moacq,\moar,\mosc\}$ denote the set 
of memory orders, where relaxed (\morlx), release (\morel),
acquire/consume (\moacq), acquire-release (\moar) and sequentially
consistent (\mosc) have been provided by C11 as memory orders that 
atomic events can be associated with for various levels
of ordering restrictions and \na corresponds to a non-atomic
memory access.
%
Let $< \subseteq \moset \times \moset$ (and accordingly $\le$) 
represent the strictness relation on memory orders such that 
$mo_1 < mo_2$ denotes that $mo_2$ is a stricter relation than 
$mo_1$ and places more restrictions on reordering of events in 
comparison to the restrictions placed by $mo_1$. 
%
The memory orders are related by the $<$ relation as $\na < \morlx
< \{\morel,\moacq\} < \moar < \mosc$, where the order \moar can only be
associated with rmw and fence events.

\noindent
{\bf Intra-thread reordering restrictions:} 
An atomic write event associated
with a memory order  $\ge \morel$ disallows events from the same 
thread occurring before it from reordering after it. 
%
Similarly, a read event with a memory order $\ge \moacq$ disallows
events occurring after it from the same thread to reorder before it.
%
Memory order \morlx does not place any restrictions on reordering.

\noindent
{\bf Inter-thread visibility restrictions:}
Memory orders may lead to ordering between events across threads. 
Based on the memory orders associated with atomic events, 
C11 defines a {\em Happens-before}(\setHB) relation on events such 
that if event A happens-before event B then memory effects of A 
become visible to B.