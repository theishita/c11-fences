As discussed in Section~\ref{sec:c11}, in a valid trace of the
input program $P$ (including a buggy trace), 
the program events must satisfy the \hlref{coherence conditions}.
%
Contrarily, if we synthesize \cc fences in the input program 
such that the irreflexivity of any of the coherence conditions 
or the total order on \sc events in the buggy trace gets violated 
then we can invalidate the trace and stop the program behavior.

Assume a set of synthesized \cc fences $\overline{\fences_{\tau'}}$
in a transformation $\tau'$ of a buggy trace $\tau$. 
%
The synthesized fences inflate the 
$\setSB$, $\setSW$ and $\setDOB$ relation sets by adding 
relations between program events ($\events_\tau$) and newly 
added fences ($\sfences_{\tau'}$) to form corresponding 
$\seqb{\tau'}{}{}$, $\sw{\tau'}{}{}$ and $\dob{\tau'}{}{}$ relations. 
%
Further, the $\ithb{\tau'}{}{}$ relation contains  
program event pairs as a consequence of the freshly formed 
$\sw{\tau'}{}{}$ and $\dob{\tau'}{}{}$ relations in addition
to the event pairs in $\setITHB$.
%
The $\mo{\tau'}{}{}$, $\rf{\tau'}{}{}$ and $\fr{\tau'}{}{}$ 
relations remain the same as the corresponding relations of 
$\tau$.

We propose two strategies, namely \stfence and \wkfence, to 
detect invalidation of the program trace $\tau$. 
%
The \wkfence strategy detects violation of a coherence condition.
If such a case exists then we can stop the behavior with a 
weaker fences (memory orders \rel, \acq, \acqrel). 
%
However, if a coherence condition is not violated then we move
to \stfence strategy to break the total order requirement on
\sc ordered events.
%
The strategies are discussed below.\newline

\noindent
{\bf Invalidate by violating coherence condition (\wkfence)}\newline
The \wkfence strategy simply attempts to detect a cycle in 
one of the \hlref{coherence conditions} (listed in 
Section~\ref{sec:c11}).
\newline

\noindent
{\bf Invalidating by violating \sc total-order (\stfence)}\newline
If the \wkfence strategy fails to violate the buggy trace then
we attempt to violate the total-order on \sc events.
%
Since the coherence condition did on violate on the events of $\tau'$ 
it implies that conditions do not violate on the \sc ordered events
of $\tau'$ either. 
%
As a result in the \stfence strategy we attempt to violate the 
irreflexivity condition on 
$(\onsc{\hb{\tau'}{}{}}$ $\union$ $\onsc{\mo{\tau'}{}{}}$ $\union$ 
$\onsc{\rf{\tau'}{}{}}$ $\union$ $\onsc{\fr{\tau'}{}{}})^+$. 

We introduce a possibly reflexive relation on \sc-ordered
events of $\tau'$, called {\em \sc-order} $(\so{\tau'}{}{})$ to 
capture the ordering between the \sc events. 
%
\begin{definition}{\bf \sc-order ($\so{\tau'}{}{}$)}\newline
	$\forall e_1, e_2 \in \events_\tau$ \st 
	$(e_1,e_2) \in$ $\setHB$ $\union$ $\setMO$ $\union$ $\setRF$ 
	$\union$ $\setFR$
	
	if
	$e_1, e_2 \in \ordevents{\sc}_\tau$ then 
	$\so{\tau'}{e_1}{e_2}$;
	
	if
	$e_1 \in \ordevents{\sc}_\tau$, 
	$\exists \mathbb{F}^\sc \in \ordfences{\sc}_{\tau'}$ where
	$\seqb{\tau'}{e_2}{\mathbb{F}^\sc}$ then
	$\so{\tau'}{e_1}{\mathbb{F}^\sc}$;
	
	if
	$e_2 \in \ordevents{\sc}_\tau$, 
	$\exists \mathbb{F}^\sc \in \ordfences{\sc}_{\tau'}$ where
	$\seqb{\tau'}{\mathbb{F}^\sc}{e_1}$ then
	$\so{\tau'}{\mathbb{F}^\sc}{e_2}$;
	
	if
	$\exists \mathbb{F}^\sc_1$, $\mathbb{F}^\sc_2$ 
	$\in \ordfences{\sc}_{\tau'}$ where
	$\seqb{\tau'}{\mathbb{F}^\sc_1}{e_1}$ and 
	$\seqb{\tau'}{e_2}{\mathbb{F}^\sc_2}$ then
	$\so{\tau'}{\mathbb{F}^\sc_1}{\mathbb{F}^\sc_1}$.
\end{definition}
%
Note that, $\setSO \subseteq \setTO$ for any trace $\tau$. It
does not contains pairs of \sc events that do not have a definite
order. Consider the example below, $\to{}{W^\sc(x,1)}{W^\sc(y,1)}$
and $\to{}{W^\sc(y,1)}{W^\sc(x,1)}$ are both valid total-orders
on the \sc events of the trace.
%
The set $\setSO$ does not contain either of the two cases and 
would be empty in this example.
%
Such a candidate pair of events cannot contribute to the 
reflexivity of $\setSO$ and can be safely ignored for the
purpose of this work. 
\begin{figure}[h]
	\resizebox{0.29\textwidth}{!}{\tikzset{every picture/.style={line width=0.75pt}} %set default line width to 0.75pt        
\begin{tikzpicture}[x=1em,y=1em,yscale=-1,xscale=-1]
\tikzstyle{every node}=[font=\normalfont]
\node (fl1) [inner sep=2pt] {$ W^\sc(x,1) 1 $};
\node (fl2) [right=35pt of fl1, inner sep=2pt] {$W^\sc(y,1)$};
\node (rfl2) [below=22pt of fl1, inner sep=2pt] {if $R^\rlx(y,0)$:};
\node (rfl1) [below=22pt of fl2, inner sep=2pt] {if $R^\rlx(x,0)$:};
%
\draw [->,>=stealth,color=CarnationPink] (fl1) -- node[midway,right=-2pt,font=\scriptsize,color=black] { $\lsb$ } (rfl2);
\draw [->,>=stealth,color=CarnationPink] (fl2) -- node[midway,left=-2pt,font=\scriptsize,color=black] { $\lsb$ } (rfl1);

\end{tikzpicture}
}
\end{figure}