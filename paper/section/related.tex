\cite{abdulla-tso-fences} solves the problem of automatic fence 
insertion in TSO programs. This technique generates counter-examples,
and identifies a set of write-read pairs, the reordering of which may 
have resulted in buggy execution. Clearly, the buggy behavior can be 
stopped by inserting fence between these pairs to stop the reorderings.
The paper finds minimal set of such pairs to stop the buggy behavior.
This technique heavily relies on the fact that only writes can reorder 
after reads. Hence, it cannot be applied for any memory model that 
allows/disallow a different set of orderings such as PSO, RMO, C11.

