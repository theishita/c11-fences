\begin{figure}[t]
	\begin{tabular}{|c||c|c|c|}
		\multicolumn{1}{c}{base rule} & 
		\multicolumn{3}{c}{extended fence rules} \\\hline
		
		\resizebox{0.24\textwidth}{!}{\tikzset{every picture/.style={line width=0.75pt}} %set default line width to 0.75pt        
\begin{tikzpicture}[x=1em,y=1em,yscale=-1,xscale=-1]
\tikzstyle{every node}=[font=\normalfont]
\node (wr1) {$ wr^\sc_1 $};
\node (wr2) [below=20pt of wr1] {$ wr^\sc_2 $};
\node (so1) [below=-5pt of wr2] {\hlref{sohb},\hlref{somo},};
\node (so2) [below left=-5pt and -66pt of so1] {\hlref{sorf},\hlref{sofr}};

\draw [->,>=stealth,color=Mahogany] ($ (wr1.south east)+(.3,-5pt) $) to[out=135,in=-135] node[midway,right=-2pt,font=\scriptsize] {\textcolor{black}{\lso}} ($ (wr2.south east)+(0.4,-7pt) $);
\draw [->,>=stealth,color=RedOrange] ($ (wr1.south west)+(-.3,-5pt) $) to[out=45,in=-45] node[left=-2pt,pos=.25,font=\scriptsize] {\textcolor{black}{\lmo/\lrf}} node[left=-0.5pt,pos=.5,font=\scriptsize] {\textcolor{black}{\lhb/\lfr}} ($ (wr2.south west)+(-0.3,-7pt) $);
\end{tikzpicture}
} &
		\resizebox{0.24\textwidth}{!}{\tikzset{every picture/.style={line width=0.75pt}} %set default line width to 0.75pt        
\begin{tikzpicture}[x=1em,y=1em,yscale=-1,xscale=-1]
\tikzstyle{every node}=[font=\normalfont]
\node (wr1) [inner sep=2pt] {$ wr^\sc_1 $};
\node (wr2) [right=25pt of wr1,inner sep=2pt] {$ wr_2 $};
\node (f1) [below left=21pt and -15pt of wr2,inner sep=2pt] {$ f^\sc_1 $};
\node (soef) [below left=0pt and -10pt of f1] {\hlref{soEF}};

`\draw [->,>=stealth,color=Mahogany] (wr1.south) -- node[midway,left=0pt,font=\scriptsize,color=black] { $\lso$ } (f1.west);
\draw [->,>=stealth,color=Cerulean] (wr1) -- node[midway,above=-2pt,font=\scriptsize,color=black] { $ \lmo/\lrf $ } node[midway,below=-2pt,font=\scriptsize,color=black] { $ \lhb/\lfr $ } (wr2);
\draw [->,>=stealth,color=CarnationPink] (wr2) -- node[midway,left=-2pt,font=\scriptsize,color=black] { $\lsb$ } (f1);

\end{tikzpicture}
} &
		\resizebox{0.24\textwidth}{!}{\tikzset{every picture/.style={line width=0.75pt}} %set default line width to 0.75pt        
\begin{tikzpicture}[x=1em,y=1em,yscale=-1,xscale=-1]
\tikzstyle{every node}=[font=\normalfont]
\node [inner sep=2pt] (f1) {$ f^\sc_1 $};
\node (wr1) [below left=20pt and -15pt of f1,inner sep=2pt] {$ wr_1 $};
\node (wr2) [right=25pt of wr1,inner sep=2pt] {$ wr^\sc_2 $};
\node (sofe) [below right=0pt and -5pt of wr1] {\hlref{soFE}};

`\draw [->,>=stealth,color=Mahogany,thin] (f1.east) -- node[midway,right=0pt,font=\scriptsize,color=black] { $\lso$ } (wr2.north);
\draw [->,>=stealth,color=Cerulean,thin] (wr1) -- node[midway,above=-2pt,font=\scriptsize,color=black] { $ \lmo/\lrf $ } node[midway,below=-2pt,font=\scriptsize,color=black] { $ \lhb/\lfr $ } (wr2);
\draw [->,>=stealth,color=CarnationPink,thin] (f1) -- node[midway,left=-2pt,font=\scriptsize,color=black] { $\lsb$ } (wr1);

\end{tikzpicture}
} &
		\resizebox{0.24\textwidth}{!}{\tikzset{every picture/.style={line width=0.75pt}} %set default line width to 0.75pt        
\begin{tikzpicture}[x=1em,y=1em,yscale=-1,xscale=-1]
\tikzstyle{every node}=[font=\normalfont]
\node (f1) [inner sep=2pt] {$ f^\sc_1 $};
\node (f2) [right=25pt of f1,inner sep=2pt] {$ f^\sc_2 $};
\node (wr1) [below left=20pt and -15pt of f1,inner sep=2pt] {$ wr_1 $};
\node (wr2) [below left=20pt and -15pt of f2,inner sep=2pt] {$ wr_2 $};
\node (soff) [below right=0pt and -5pt of wr1] {\hlref{soFF}};

`\draw [->,>=stealth,color=Mahogany,thin] (f1) -- node[midway,above=-2pt,font=\scriptsize,color=black] { $\lso$ } (f2);
\draw [->,>=stealth,color=Cerulean,thin] (wr1) -- node[midway,above=-2pt,font=\scriptsize,color=black]{\lmo/} node[midway,below=-2pt,font=\scriptsize,color=black]{ $ \lhb/\lrf $ } (wr2);
\draw [->,>=stealth,color=CarnationPink,thin] (f1) -- node[midway,left=-2pt,font=\scriptsize,color=black] { $\lsb$ } (wr1);
\draw [->,>=stealth,color=CarnationPink,thin] (wr2) -- node[midway,left=-2pt,font=\scriptsize,color=black] { $\lsb$ } (f2);

%\draw [->,>=stealth,color=orange,thin] ($ (ew1.south east)+(.5,-5pt) $) to[out=135,in=-135] node[midway,right=-2pt,font=\scriptsize] {mo} ($ (ew2.south east)+(0.4,-5pt) $);
%\draw [->,>=stealth,color=red,thin] ($ (ew1.south west)+(-.3,-5pt) $) to[out=45,in=-45] node[midway,left=-2pt,font=\scriptsize] {c::hb} ($ (ew2.south west)+(-0.3,-5pt) $);


\end{tikzpicture}
} \\
		\hline
		\multicolumn{4}{c}{(a) \lso-rules} \\
		\hline
		
		\resizebox{0.24\textwidth}{!}{\tikzset{every picture/.style={line width=0.75pt}} %set default line width to 0.75pt        
\begin{tikzpicture}[x=1em,y=1em,yscale=-1,xscale=-1]
\tikzstyle{every node}=[font=\normalfont]
\node (w1) {$ w^\rel_1 $};
\node (r1) [below=20pt of w1] {$ r^\acq_1 $};
\node (sw) [below=-5pt of r1] {\hlref{sw}};

\draw [->,>=stealth,color=Magenta,thin] ($ (w1.south east)+(.3,-5pt) $) to[out=135,in=-135] node[midway,right=-2pt,font=\scriptsize] {\textcolor{black}{\lsw}} ($ (r1.south east)+(0.4,-7pt) $);
\draw [->,>=stealth,color=PineGreen,thin] ($ (w1.south west)+(-.3,-5pt) $) to[out=45,in=-45] node[left=-2pt,pos=.25,font=\scriptsize] {\textcolor{black}{\lrf}}($ (r1.south west)+(-0.3,-7pt) $);

\end{tikzpicture}
} &
		\resizebox{0.24\textwidth}{!}{\tikzset{every picture/.style={line width=0.75pt}} %set default line width to 0.75pt        
\begin{tikzpicture}[x=1em,y=1em,yscale=-1,xscale=-1]
\tikzstyle{every node}=[font=\normalfont]
\node (w1) [inner sep=2pt] {$ w^\rel_1 $};
\node (r1) [right=25pt of w1,inner sep=2pt] {$ r_1 $};
\node (f1) [below left=19pt and -15pt of r1,inner sep=2pt] {$ f^\acq_1 $};
\node (swef) [below left=-3pt and -10pt of f1] {\hlref{swEF}};

`\draw [->,>=stealth,color=Magenta,thin] (w1.south) -- node[midway,left=0pt,font=\scriptsize,color=black] { $\lsw$ } (f1.west);
\draw [->,>=stealth,color=PineGreen,thin] (w1) -- node[midway,above=-2pt,font=\scriptsize,color=black] { $ \lrf $ }  (r1);
\draw [->,>=stealth,color=CarnationPink,thin] (r1) -- node[midway,left=-2pt,font=\scriptsize,color=black] { $\lsb$ } (f1);

\end{tikzpicture}
} &
		\resizebox{0.24\textwidth}{!}{\tikzset{every picture/.style={line width=0.75pt}} %set default line width to 0.75pt        
\begin{tikzpicture}[x=1em,y=1em,yscale=-1,xscale=-1]
\tikzstyle{every node}=[font=\normalfont]
\node [inner sep=2pt] (f1) {$ f^\rel_1 $};
\node (w1) [below left=21pt and -15pt of f1,inner sep=2pt] {$ w_1 $};
\node (r1) [right=25pt of w1,inner sep=2pt] {$ r^\acq_1 $};
\node (swfe) [below right=0pt and -5pt of w1] {\hlref{swFE}};

`\draw [->,>=stealth,color=Magenta] (f1.east) -- node[midway,right=0pt,font=\scriptsize,color=black] { $\lsw$ } (r1.north);
\draw [->,>=stealth,color=PineGreen] (w1) -- node[midway,above=-2pt,font=\scriptsize,color=black] { $ \lrf $ } (r1);
\draw [->,>=stealth,color=CarnationPink] (f1) -- node[midway,left=-2pt,font=\scriptsize,color=black] { $\lsb$ } (w1);

\end{tikzpicture}
} &
		\resizebox{0.24\textwidth}{!}{\tikzset{every picture/.style={line width=0.75pt}} %set default line width to 0.75pt        
\begin{tikzpicture}[x=1em,y=1em,yscale=-1,xscale=-1]
\tikzstyle{every node}=[font=\normalfont]
\node (f1) [inner sep=2pt] {$ f^\rel_1 $};
\node (f2) [right=25pt of f1,inner sep=2pt] {$ f^\acq_2 $};
\node (w1) [below left=20pt and -15pt of f1,inner sep=2pt] {$ w_1 $};
\node (r1) [below left=20pt and -15pt of f2,inner sep=2pt] {$ r_1 $};
\node (swff) [below right=-2pt and -5pt of wr1] {\hlref{swFF}};

`\draw [->,>=stealth,color=Magenta] (f1) -- node[midway,above=-2pt,font=\scriptsize,color=black] { $\lsw$ } (f2);
\draw [->,>=stealth,color=PineGreen] (w1) -- node[midway,above=-2pt,font=\scriptsize,color=black]{\lrf} (r1);
\draw [->,>=stealth,color=CarnationPink] (f1) -- node[midway,left=-2pt,font=\scriptsize,color=black] { $\lsb$ } (w1);
\draw [->,>=stealth,color=CarnationPink] (r1) -- node[midway,left=-2pt,font=\scriptsize,color=black] { $\lsb$ } (f2);

%\draw [->,>=stealth,color=orange] ($ (ew1.south east)+(.5,-5pt) $) to[out=135,in=-135] node[midway,right=-2pt,font=\scriptsize] {mo} ($ (ew2.south east)+(0.4,-5pt) $);
%\draw [->,>=stealth,color=red] ($ (ew1.south west)+(-.3,-5pt) $) to[out=45,in=-45] node[midway,left=-2pt,font=\scriptsize] {c::hb} ($ (ew2.south west)+(-0.3,-5pt) $);


\end{tikzpicture}
} \\
		
		\resizebox{0.24\textwidth}{!}{\tikzset{every picture/.style={line width=0.75pt}} %set default line width to 0.75pt        
\begin{tikzpicture}[x=1em,y=1em,yscale=-1,xscale=-1]
\tikzstyle{every node}=[font=\normalfont]
\node (w1) [inner sep=2pt] {$ w^\rel_1 $};
\node (r1) [right=25pt of w1,inner sep=2pt] {$ r^\acq_1 $};
\node (w2) [below left=21pt and -15pt of w1,inner sep=2pt] {$ w_2 $};
\node (dob) [below right=0pt and -5pt of w2] {\hlref{dob}};

`\draw [->,>=stealth,color=Mulberry] (w1.east) -- node[midway,above=-2pt,font=\scriptsize,color=black] { $\ldob$ } (r1.west);
\draw [->,>=stealth,color=PineGreen] (w2) -- node[midway,below=-2pt,font=\scriptsize,color=black] { $ \lrf $ }  (r1);
\draw [->,>=stealth,color=CarnationPink] (w1) -- node[midway,left=-2pt,font=\scriptsize,color=black] { $\lsb$ } (w2);

\end{tikzpicture}
} &
		\resizebox{0.24\textwidth}{!}{\tikzset{every picture/.style={line width=0.75pt}} %set default line width to 0.75pt        
\begin{tikzpicture}[x=1em,y=1em,yscale=-1,xscale=-1]
\tikzstyle{every node}=[font=\normalfont]
\node (w1) [inner sep=2pt] {$ w^\rel_1 $};
\node (f1) [right=25pt of w1,inner sep=2pt] {$ f^\acq_1 $};
\node (w2) [below left=20pt and -15pt of w1,inner sep=2pt] {$ w_2 $};
\node (r1) [below left=20pt and -13pt of f1,inner sep=2pt] {$ r_1 $};
\node (dobEF) [below right=0pt and -5pt of w2] {\hlref{dobEF}};

`\draw [->,>=stealth,color=PineGreen] (w2) -- node[midway,above=-2pt,font=\scriptsize,color=black] { $\lrf$ } (r1);
\draw [->,>=stealth,color=CarnationPink] (w1) -- node[midway,left=-2pt,font=\scriptsize,color=black] { $ \lsb $ } (w2);
\draw [->,>=stealth,color=CarnationPink] (r1) -- node[midway,left=-2pt,font=\scriptsize,color=black] { $\lsb$ } (f1);
\draw [->,>=stealth,color=Mulberry] (w1) -- node[midway,above=-2pt,font=\scriptsize,color=black] { $\ldob$ } (f1);

\end{tikzpicture}
} &&\\
		\hline
		\multicolumn{4}{c}{(b) \lsw- and \ldob-rules} \\
		\hline
		
		\resizebox{0.24\textwidth}{!}{\tikzset{every picture/.style={line width=0.75pt}} %set default line width to 0.75pt        
\begin{tikzpicture}[x=1em,y=1em,yscale=-1,xscale=-1]
\tikzstyle{every node}=[font=\normalfont]
\node (r1) [inner sep=2pt] {$ r^\acq_1 $};
\node (w1) [right=25pt of r1,inner sep=2pt] {$ w^\rel_1 $};
\node (r2) [below left=25pt and -13pt of r1, inner sep=1pt] {$ r_2 $};
\node (w2) [below left=25pt and -15pt of w1, inner sep=1pt] {$ w_2 $};
%\node (w3) [below right=2pt and 1pt of r1,inner sep=2pt] {$ w_3 $};
\node (ws) [below right=0pt and 5pt of r2] {\hlref{ws}};

%`\draw [->,>=stealth,color=RedOrange,thin] (w3) -- node[pos=0.7,left=-2pt,font=\scriptsize,color=black] { $\lmo$ } (w2);
%\draw [->,>=stealth,color=PineGreen,thin] (w3) -- node[pos=0.7,right=-2pt,font=\scriptsize,color=black] { $ \lrf $ } (r2);
\draw [->,>=stealth,color=CarnationPink,thin] (r1) -- node[midway,left=-2pt,font=\scriptsize,color=black] { $\lsb$ } (r2);
\draw [->,>=stealth,color=CarnationPink,thin] (w2) -- node[midway,left=-2pt,font=\scriptsize,color=black] { $\lsb$ } (w1);
\draw [->,>=stealth,color=Mahogany,thin] (r1) -- node[midway,above=-2pt,font=\scriptsize,color=black] { $\lws$ } (w1);
\draw [->,>=stealth,color=RoyalPurple,thin] (r2) -- node[midway,above=-2pt,font=\scriptsize,color=black] { $\lfr$ } (w2);

\end{tikzpicture}
} &
		\resizebox{0.24\textwidth}{!}{\tikzset{every picture/.style={line width=0.75pt}} %set default line width to 0.75pt        
\begin{tikzpicture}[x=1em,y=1em,yscale=-1,xscale=-1]
\tikzstyle{every node}=[font=\normalfont]
\node (r1) [inner sep=2pt] {$ r^\acq_1 $};
\node (f1) [right=25pt of r1,inner sep=2pt] {$ f^\rel_1 $};
\node (r2) [below left=25pt and -13pt of r1, inner sep=1pt] {$ r_2 $};
\node (w2) [below left=25pt and -15pt of f1, inner sep=1pt] {$ w_1 $};
%\node (w1) [below right=2pt and 1pt of r1,inner sep=2pt] {$ w_1 $};
\node (wsEF) [below right=0pt and 0pt of r2] {\hlref{wsEF}};

\draw [->,>=stealth,color=CarnationPink] (r1) -- node[midway,left=-2pt,font=\scriptsize,color=black] { $\lsb$ } (r2);
\draw [->,>=stealth,color=CarnationPink] (w2) -- node[midway,left=-2pt,font=\scriptsize,color=black] { $\lsb$ } (f1);
\draw [->,>=stealth,color=Mahogany] (r1) -- node[midway,above=-2pt,font=\scriptsize,color=black] { $\lws$ } (f1);
\draw [->,>=stealth,color=RoyalPurple] (r2) -- node[midway,above=-2pt,font=\scriptsize,color=black] { $\lfr$ } (w2);

\end{tikzpicture}
} & 
		\resizebox{0.24\textwidth}{!}{\tikzset{every picture/.style={line width=0.75pt}} %set default line width to 0.75pt        
\begin{tikzpicture}[x=1em,y=1em,yscale=-1,xscale=-1]
\tikzstyle{every node}=[font=\normalfont]
\node (f1) [inner sep=2pt] {$ f^\acq_1 $};
\node (wx) [right=25pt of f1,inner sep=2pt] {$ w^\rel_1 $};
\node (r1) [below left=25pt and -13pt of f1, inner sep=1pt] {$ r_1 $};
\node (w2) [below left=25pt and -15pt of wx, inner sep=1pt] {$ w_2 $};
%\node (w1) [below right=2pt and 1pt of f1,inner sep=2pt] {$ w_2 $};
\node (wsFE) [below right=0pt and 5pt of r1] {\hlref{wsFE}};

\draw [->,>=stealth,color=CarnationPink] (f1) -- node[midway,left=-2pt,font=\scriptsize,color=black] { $\lsb$ } (r1);
\draw [->,>=stealth,color=CarnationPink] (w2) -- node[midway,left=-2pt,font=\scriptsize,color=black] { $\lsb$ } (wx);
\draw [->,>=stealth,color=Mahogany] (f1) -- node[midway,above=-2pt,font=\scriptsize,color=black] { $\lws$ } (wx);
\draw [->,>=stealth,color=RoyalPurple] (r2) -- node[midway,above=-2pt,font=\scriptsize,color=black] { $\lfr$ } (w2);

\end{tikzpicture}
} &
		\resizebox{0.24\textwidth}{!}{\tikzset{every picture/.style={line width=0.75pt}} %set default line width to 0.75pt        
\begin{tikzpicture}[x=1em,y=1em,yscale=-1,xscale=-1]
	\tikzstyle{every node}=[font=\normalfont]
	\node (f1) [inner sep=2pt] {$ f^\acq_1 $};
	\node (f2) [right=25pt of f1,inner sep=2pt] {$ f^\rel_2 $};
	\node (r1) [below left=25pt and -13pt of f1, inner sep=1pt] {$ r_1 $};
	\node (w1) [below left=25pt and -15pt of wx, inner sep=1pt] {$ w_1 $};
	%\node (w1) [below right=2pt and 1pt of f1,inner sep=2pt] {$ w_2 $};
	\node (wsFF) [below right=0pt and 5pt of r1] {\hlref{wsFF}};
	
	%`\draw [->,>=stealth,color=RedOrange,thin] (w1) -- node[pos=0.7,left=-2pt,font=\scriptsize,color=black] { $\lmo$ } (w2);
	%\draw [->,>=stealth,color=PineGreen,thin] (w1) -- node[pos=0.7,right=-2pt,font=\scriptsize,color=black] { $ \lrf $ } (r1);
	\draw [->,>=stealth,color=CarnationPink,thin] (f1) -- node[midway,left=-2pt,font=\scriptsize,color=black] { $\lsb$ } (r1);
	\draw [->,>=stealth,color=CarnationPink,thin] (w2) -- node[midway,left=-2pt,font=\scriptsize,color=black] { $\lsb$ } (wx);
	\draw [->,>=stealth,color=Mahogany,thin] (f1) -- node[midway,above=-2pt,font=\scriptsize,color=black] { $\lws$ } (wx);
	\draw [->,>=stealth,color=RoyalPurple,thin] (r2) -- node[midway,above=-2pt,font=\scriptsize,color=black] { $\lfr$ } (w2);
	
\end{tikzpicture}
} \\
		\hline
		\multicolumn{4}{c}{(c) \lws-rules} \\	
	\end{tabular}
	\caption{Rules to form relation between program events and 
		candidate fences}
	\label{fig:so rules}
\end{figure}

\ourtechnique attempts to invalidate a buggy trace (\aka counter
example) by one of the following two strategies:
\begin{itemize}[label=strategy1,align=left,leftmargin=*]
	\item [Strategy1]:
		Invalidate by violating \sc-total-order.
	\item [Strategy2]:
		Invalidate by inter-thread synchronization.
\end{itemize}

\noindent
{\bf Strategy1: Invalidate by violating \sc-total-order. 
	(\sfence)}\newline
Consider a buggy input program $P$.
%
As discussed in Section~\ref{sec:c11}, in a valid trace of $P$ 
(including a buggy trace), 
the \sc ordered events must form a total order ($\setTO$).
%
Contrarily, if we synthesize \sc ordered fences in the input program 
such that the total order requirement in the buggy trace gets violated 
then we can invalidate the trace and stop the program behavior.

We introduce an irreflexive and possibly cyclic relation
on \sc ordered events called \sc-order ($\setSO$).
%
To construct $\setSO$, we introduce a set of \lso-rules
(diagrammatically shown in Figure~\ref{fig:so rules}(a) and
discussed below).
%
The \lso base rules are simply implied from
the \lto-rules, \hlref{toHb}, \hlref{toMo} \hlref{toFr} 
and \hlref{toRf} (Section~\ref{sec:c11}).
%
The extended rules apply respective $\lto$-rules to relaxed
events with appropriately placed fences as described in
\cc \cite{C11}\cite{Batty-POPL12}.

\begin{longtable}{|p{0.11\textwidth} p{0.88\textwidth}|}
	\hline
	\multicolumn{2}{|l|}{\bf \lso-base rules:}\\
	
	\hl{sohb}, & 
	$\forall wr^{\sc}_1, wr^{\sc}_2 \in \ordevents{\sc}_\tau$ if \\
	\hl{somo}, &
	$\hb{\tau}{wr^{\sc}_1}{wr^{\sc}_2}$ $\v$ $\mo{\tau}{wr^{\sc}_1}{wr^{\sc}_2}$
	$\v$ $\rf{\tau}{wr^{\sc}_1}{wr^{\sc}_2}$ 
	$\v$ $\fr{\tau}{wr^{\sc}_1}{wr^{\sc}_2}$\\
	\hl{sorf}, & 
	then $\so{\tau}{wr^{\sc}_1}{wr^{\sc}_2}$ \\
	\hl{sofr}: & ($\setHB$, $\setMO$, $\setRF$. $\setFR$ between \sc events 
				implies $\setSO$) \\
	& \\
	
	\multicolumn{2}{|l|}{\bf \lso-rules extended for fences:} \\
	
	\hl{soEF}: & $\forall wr^\sc_1 \in \ordevents{\sc}_\tau$, $f^\sc_1 \in
	\ordfences{\sc}_\tau$ if $\exists wr_2 \in \events_\tau$ \st 
	($\mo{\tau}{wr^\sc_1}{wr_2}$ $\v$ $\hb{\tau}{wr^\sc_1}{wr_2}$
	$\v$ $\rf{\tau}{wr^\sc_1}{wr_2}$) $\^$ $\seqb{\tau}{wr_2}{f^\sc_1}$ 
	then $\so{\tau}{wr^\sc_1}{f^\sc_1}$ \\
	& ($\setHB$, $\setMO$, $\setRF$, $\setFR$ between \sc event and relaxed 
		event forms $\setSO$ with assistance of an appropriately placed 
		fence) \\
	
	\hl{soFE}: & $\forall f^\sc_1 \in \ordfences{\sc}_\tau$, $wr^\sc_2 \in
	\ordevents{\sc}_\tau$ if $\exists wr_1 \in \events_\tau$ \st 
	($\mo{\tau}{wr_1}{wr^\sc_2}$ $\v$ $\hb{\tau}{wr_1}{wr^\sc_2}$
	$\v$ $\rf{\tau}{wr_1}{wr^\sc_2}$) $\^$ $\seqb{\tau}{f^\sc_1}{wr_1}$ 
	then $\so{\tau}{f^\sc_1}{wr^\sc_2}$ \\
	& ($\setHB$, $\setMO$, $\setRF$, $\setFR$ between relaxed event and
	\sc event forms $\setSO$ with assistance of an appropriately placed 
	fence) \\
	
	\hl{soFF}: & $\forall f^{\sc}_1, f^{\sc}_2 \in \ordfences{\sc}_\tau$, 
	if $\exists wr_1, wr_2 \in \events_\tau$ \st 
	($\mo{\tau}{wr_1}{wr_2}$ $\v$ $\hb{\tau}{wr_1}{wr_2}$
	$\v$ $\rf{\tau}{wr_1}{wr_2}$) $\^$
	($\seqb{\tau}{f^{\sc}_1}{wr_1}$, $\seqb{\tau}{wr_2}{f^{\sc}_2}$) 
	then $\so{\tau}{f^{\sc}_1}{f^{\sc}_2}$ \\
	& ($\setHB$, $\setMO$, $\setRF$, $\setFR$ between relaxed events 
		forms $\setSO$ with assistance of appropriately placed fences) \\
	\hline
\end{longtable}
%
Consider a \cc trace $\tau$ and a transformation
$\inv{\tau}$ of $\tau$ \st $\events_{\inv{\tau}}$ = $\events_\tau$
$\union$ set of synthesized fences, then;
%
\begin{theorem} \label{thm:to-so}
	$\to{\inv{\tau}}{}{}$ = (transitive closure of $\so{\inv{\tau}}{}{}$).
\end{theorem}
%
Using Theorem~\ref{thm:to-so} we can state that,
a cyclic $\so{\inv{\tau}}{}{}$ implies that there does not exist a 
total order on \sc ordered events of $\inv{\tau}$. The trace $\inv{\tau}$
then is not a valid \cc trace and the buggy trace $\tau$ 
has been invalidated.
%
Thus, the aim of \sfence is to synthesis \sc-fences in the input program $P$
at appropriate locations that would force a cyclic $\so{\inv{\tau}}{}{}$ 
 in the \sc ordered events of a transformed trace $\inv{\tau}$. 
The cycle invalidates the trace for the transformed program $\fx{P}$.

\setlength{\textfloatsep}{0pt}
\begin{wrapfigure}{l}{0.5\textwidth}
	\vspace{-2.5em}
	\begin{tabular}{|c|}
		\multicolumn{1}{c}{Initially $Flag_1 = 0, Flag_2 = 0$} \\
		\hline
		\resizebox{0.45\textwidth}{!}{\tikzset{every picture/.style={line width=0.75pt}} %set default line width to 0.75pt        
\begin{tikzpicture}[x=1em,y=1em,yscale=-1,xscale=-1]
\tikzstyle{every node}=[font=\normalfont]
\node (fl1) [inner sep=2pt] {$ Flag_1 :=_\sc 1 $};
\node (fl2) [right=35pt of fl1, inner sep=2pt] {$Flag_2 :=_\sc 1$};
\node (rfl2) [below=52pt of fl1, inner sep=2pt] {if $\neg Flag_2$:};
\node (rfl1) [below=52pt of fl2, inner sep=2pt] {if $\neg Flag_1$:};
\node (cs11) [below right=-1pt and -40pt of rfl2] {(critical};
\node (cs12) [below right=-4pt and -40pt of cs11] {section1)};
\node (cs21) [below right=-1pt and -40pt of rfl1] {(critical};
\node (cs22) [below right=-4pt and -40pt of cs21] {section2)};
\node (t) [below right=0pt and -30pt of cs12] {(i) {\tt Buggy trace ($\tau$)}};
%
\draw [->,>=stealth,color=CarnationPink] (fl1) -- node[midway,right=-2pt,font=\scriptsize,color=black] { $\lsb$ } (rfl2);
\draw [->,>=stealth,color=CarnationPink] (fl2) -- node[midway,left=-2pt,font=\scriptsize,color=black] { $\lsb$ } (rfl1);

\end{tikzpicture}
} \\
		\hline
		\resizebox{0.49\textwidth}{!}{\tikzset{every picture/.style={line width=0.75pt}} %set default line width to 0.75pt        
\begin{tikzpicture}[x=1em,y=1em,yscale=-1,xscale=-1]
\tikzstyle{every node}=[font=\normalfont]
\node (fl1) [inner sep=2pt] {$ Flag_1 :=_\sc 1 $};
\node (fl2) [right=35pt of fl1, inner sep=2pt] {$Flag_2 :=_\sc 1$};
\node (f1) [below=20pt of fl1, inner sep=2pt] {$\mathbb{F}^\sc_1$};
\node (f2) [below=20pt of fl2, inner sep=2pt] {$\mathbb{F}^\sc_2$};
\node (rfl2) [below=20pt of f1, inner sep=2pt] {if $\neg Flag_2$:};
\node (rfl1) [below=20pt of f2, inner sep=2pt] {if $\neg Flag_1$:};
\node (cs11) [below right=-1pt and -40pt of rfl2] {(critical};
\node (cs12) [below right=-4pt and -40pt of cs11] {section1)};
\node (cs21) [below right=-1pt and -40pt of rfl1] {(critical};
\node (cs22) [below right=-4pt and -40pt of cs21] {section2)};
\node (t) [below right=0pt and -35pt of cs12] {(ii) {\tt Invalidated ($\inv{\tau}$)}};
%
\draw [->,>=stealth,color=Mahogany] (fl1) -- node[midway,right=-2pt,font=\scriptsize,color=black] { $\lsb$ } node[pos=.25,left=-2pt,font=\scriptsize,color=black] { $\lso$ } node[pos=.65,left=-2pt,font=\scriptsize,color=black] { \hlref{sohb} } (f1);
\draw [->,>=stealth,color=Mahogany] (fl2) -- node[midway,left=-2pt,font=\scriptsize,color=black] { $\lsb$ } node[pos=.25,right=-2pt,font=\scriptsize,color=black] { $\lso$ } node[pos=.65,right=-2pt,font=\scriptsize,color=black] { \hlref{sohb} } (f2);
\draw [->,>=stealth,color=CarnationPink] (f1) -- node[midway,right=-2pt,font=\scriptsize,color=black] { $\lsb$ } (rfl2);
\draw [->,>=stealth,color=CarnationPink] (f2) -- node[midway,left=-2pt,font=\scriptsize,color=black] { $\lsb$ } (rfl1);
\draw [->,>=stealth,color=Mahogany] (f2) -- node[pos=.25,sloped,above=-2pt,font=\scriptsize,color=black] { $\lso$ } node[pos=.25,sloped,below=-2pt,font=\scriptsize,color=black] { \hlref{soFE} } (fl1);
\draw [->,>=stealth,color=Mahogany] (f1) -- node[pos=.25,sloped,above=-2pt,font=\scriptsize,color=black] { $\lso$ } node[pos=.25,sloped,below=-2pt,font=\scriptsize,color=black] { \hlref{soFE} } (fl2);

\end{tikzpicture}
} \\
		\hline
		\multicolumn{1}{c}{\hl{mutex}}
	\end{tabular}
	\vspace{-2.5em}
\end{wrapfigure}

Consider the example \hlref{mutex}. If the reads of $Flag_1$ and $Flag_2$ 
read the initial value $0$ (as shown in \hlref{mutex}(i)) 
then the program violates mutual exclusion
property. Note that, the relation $\setFR$ is not a \cc relation, it has
been introduced in this work to assist with forming $\setSO$ relation.
In the absence of $\setFR$, the trace $\tau$ does not violate any coherence
rule and is possible under \cc.
%
The buggy trace can be invalidated by introducing \sc-fences $\mathbb{F}^\sc_1$
and $\mathbb{F}^\sc_2$ as shown in \hlref{mutex}(ii). The \lso-rules
\hlref{sohb} and \hlref{soFE} 
%(as $\setSB{\implies}\setSO$ and 
%$\setFR{\implies}\setSO$) 
construct the $\so{\inv{\tau}}{}{}$ edges between the program events 
and synthesized fences to form a cyclic $\so{\inv{\tau}}{}{}$ relation 
shown in \hlref{mutex}(ii).
The cycle in $\so{\inv{\tau}}{}{}$ relation indicates that a valid 
total-order cannot be formed on \sc-events of $\inv{\tau}$ and thus the 
trace cannot be produced under \cc.\newline

\noindent
{\bf Strategy2: Invalidate by inter-thread synchronization.
(\wfence)}\newline
In the second strategy we attempt to introduce synchronization
between threads by synthesizing fences. The fences form
$\setHB$ relation with program events \st coherence
requirement
$\neg(\hb{\tau}{e_1}{e_2}$ $\^$ $\fr{\tau}{e_2}{e_1})$ is
violated.
%
Note that, this strategy invalidates a buggy trace by introducing 
weaker fences (of memory orders \rel, \acq and \acqrel) and 
is preferred over Strategy1 wherever applicable.

Two threads synchronize over \rel and \acq (or stricter) ordered
events when an appropriate $\setRF$ is formed. Such synchronization
form $\setSW$ ({\it synchronizes-with}) and $\setDOB$ 
({\it dependency-ordered-before}) relations between program events,
\cite{Batty-POPL12}\cite{C11}
where $\setSW \subseteq \setITHB$ and $\setDOB \subseteq \setITHB$.
%
The $\setSW$ and $\setDOB$ relations can similarly be formed between
fences based on appropriate $\setRF$. 
The rules have been formally presented 
below and diagrammatically represented in Fig~\ref{fig:so rules}(ii).

\begin{longtable}{|p{0.11\textwidth} p{0.88\textwidth}|}
	\hline
	\multicolumn{2}{|l|}{\bf \lsw-, \ldob-base rule:}\\
	
	\hl{sw}: &
	$\forall$ $w^\rel_1 \in \ordwrites{\moge\rel}_\tau$ and
	$r^\acq_1 \in \ordreads{\moge\acq}_\tau$ 
	if $\rf{\tau}{w^\rel_1}{r^\acq_1}$ then
	$\sw{\tau}{w^\rel_1}{r^\acq_1}$ \\
	& ($\setRF$ between $\moge\rel$ and $\moge\acq$ events 
		forms $\setSW$) \\
	
	\hl{dob}: &
	$\forall$ $w^\rel_1 \in \ordwrites{\moge\rel}_\tau$,
	$r^\acq_1 \in \ordreads{\moge\acq}_\tau$ and
	$w_2 \in$ {\it release-sequence}($w^\rel_1$) \cite{C11}
	if $\rf{\tau}{w_2}{r^\acq_1}$ then
	$\dob{\tau}{w^\rel_1}{r^\acq_1}$ \\
	& ($\setRF$ between release-sequence of $\moge\rel$  
		and $\moge\acq$ events forms $\setDOB$) \\
		
	&\\
	\multicolumn{2}{|l|}{\bf \lsw-, \ldob-rules extended for fences}\\
	
	\hl{swEF}: &
	$\forall$ $w^\rel_1 \in \ordwrites{\moge\rel}_\tau$,
	$r_1 \in \reads_\tau$ and $f^\acq_1 \in \ordfences{\moge\acq}_\tau$
	\st $\seqb{\tau}{r_1}{f^\acq_1}$
	if $\rf{\tau}{w^\rel_1}{r_1}$ then
	$\sw{\tau}{w^\rel_1}{f^\acq_1}$ \\
	& ($\setRF$ between $\moge\rel$ event and relaxed event forms 
		$\setSW$ with assistance of an appropriately placed 
		fence) \\
		
	\hl{swFE}: &
	$\forall$ $w_1 \in \writes_\tau$,
	$r^\acq_1 \in \ordreads{\moge\acq}_\tau$ and 
	$f^\rel_1 \in \ordfences{\moge\rel}_\tau$
	\st $\seqb{\tau}{f^\rel_1}{w_1}$
	if $\rf{\tau}{w_1}{r^\acq_1}$ then
	$\sw{\tau}{f^\rel_1}{r^\acq_1}$ \\
	& ($\setRF$ between relaxed event and $\moge\acq$ event forms 
	$\setSW$ with assistance of an appropriately placed 
	fence) \\
	
	\hl{swFF}: &
	$\forall$ $w_1 \in \writes_\tau$, $r_1 \in \reads_\tau$,
	$f^\rel_1 \in \ordfences{\moge\rel}_\tau$ and
	$f^\acq_2 \in \ordfences{\moge\acq}_\tau$ \st
	$\seqb{\tau}{f^\rel_1}{w_1}$, $\seqb{\tau}{r_1}{f^\acq_2}$
	if $\rf{\tau}{w_1}{r_1}$ then
	$\sw{\tau}{f^\rel_1}{f^\acq_2}$ \\
	& ($\setRF$ between relaxed events forms 
	$\setSW$ with assistance of appropriately placed 
	fences) \\
	
	\hl{dobEF}: &
	$\forall$ $w^\rel_1 \in \ordwrites{\moge\rel}_\tau$, 
	$w_2 \in \writes_\tau$, $r_1 \in \reads_\tau$ and
	$f^\acq_1 \in \ordfences{\moge\acq}_\tau$ \st
	$\seqb{\tau}{w^\rel_1}{w_2}$, $\seqb{\tau}{r_1}{f^\acq_1}$
	if $\rf{\tau}{w_2}{r_1}$ then
	$\sw{\tau}{w^\rel_1}{f^\acq_1}$ \\
	& ($\setRF$ between release-sequence of $\moge\rel$  
	and relaxed events forms $\setDOB$ with assistance of an 
	appropriately placed fence) \\
	
	\hline
	\multicolumn{2}{r}{\scriptsize where, 
		$\moge\rel$ = $\{\rel,\acqrel,\sc\}$, 
		$\moge\acq$ = $\{\acq,\acqrel,\sc\}$}
\end{longtable}

\setlength{\textfloatsep}{0pt}
\begin{wrapfigure}{l}{0.6\textwidth}
	\vspace{-2.5em}
	\begin{tabular}{|c|c|}
		\multicolumn{2}{c}{Initially $x = 0, y = 0$} \\
		
		\hline
		\resizebox{0.29\textwidth}{!}{\tikzset{every picture/.style={line width=0.75pt}} %set default line width to 0.75pt        
\begin{tikzpicture}[x=1em,y=1em,yscale=-1,xscale=-1]
\tikzstyle{every node}=[font=\normalfont]
\node (wy) [inner sep=2pt] {$W^\rlx(y,1)$};
\node (rx) [right=10pt of wy, inner sep=2pt] {$R^\rlx(x,1)$};
\node (wx) [below=52pt of wy, inner sep=2pt] {$W^\rlx(x,1)$};
\node (ry) [below=52pt of rx, inner sep=2pt] {$R^\rlx(y,0)$};
\node (t) [below right=1pt and -45pt of wx] {(i) {\tt Buggy trace ($\tau$)}};
%
\draw [->,>=stealth,color=CarnationPink] (wy) -- node[midway,right=-2pt,font=\small,color=black] { $\lsb$ } (wx);
\draw [->,>=stealth,color=CarnationPink] (rx) -- node[midway,left=-2pt,font=\small,color=black] { $\lsb$ } (ry);
\draw [->,>=stealth,color=PineGreen] (wx) -- node[pos=.75,left,font=\small,color=black] { $\lrf$ } (rx);

\end{tikzpicture}
} &
		\resizebox{0.29\textwidth}{!}{\tikzset{every picture/.style={line width=0.75pt}} %set default line width to 0.75pt        
\begin{tikzpicture}[x=1em,y=1em,yscale=-1,xscale=-1]
\tikzstyle{every node}=[font=\normalfont]
\node (wy) [inner sep=2pt] {$W^\rlx(y,1)$};
\node (rx) [right=10pt of wy, inner sep=2pt] {$R^\rlx(x,1)$};
\node (f1) [below=20pt of wy, inner sep=2pt] {$\mathbb{F}^\rel_1$};
\node (f2) [below=20pt of rx, inner sep=2pt] {$\mathbb{F}^\acq_2$};
\node (wx) [below=20pt of f1, inner sep=2pt] {$W^\rlx(x,1)$};
\node (ry) [below=20pt of f2, inner sep=2pt] {$R^\rlx(y,0)$};
\node (t) [below right=-1pt and -42pt of wx] {(ii) {\tt Invalid ($\inv{\tau}$)}};
%
\draw [->,>=stealth,color=Mahogany] (wy) -- node[midway,left=-2pt,font=\small,color=black] { $\lsb$ } (f1);
\draw [->,>=stealth,color=Mahogany] (rx) -- node[midway,right=-2pt,font=\small,color=black] { $\lsb$ } (f2);
\draw [->,>=stealth,color=Mahogany] (f1) -- node[midway,left=-2pt,font=\small,color=black] { $\lsb$ } (wx);
\draw [->,>=stealth,color=Mahogany] (f2) -- node[midway,right=-2pt,font=\small,color=black] { $\lsb$ } (ry);
\draw [->,>=stealth,color=PineGreen,opacity=0.5] (wx) -- node[midway,left=-2pt,font=\small,color=black] { $\lrf$ } (rx);
\draw [->,>=stealth,color=Magenta] (f1.north east) to[out=-165,in=-15] node[midway,above=5pt,font=\small,color=black] { $\lsw$ } node[midway,above=-1pt,font=\small,color=black] {\tt \textcolor{Sepia}{(sw-dobFF)} } (f2.north west);
\draw [->,>=stealth,color=Mahogany] (f2.south west) to[out=15,in=165] node[midway,below=0pt,font=\small,color=black] { $\lws$ } node[midway,below=5pt,font=\small,color=black] {\tt \textcolor{Sepia}{(wsFF)} } (f1.south east);

%(wr1.south east)+(.3,-5pt) $) to[out=135,in=-135]

\end{tikzpicture}
} \\
		\hline

		\multicolumn{2}{r}{\scriptsize 
		$W/R^m(o,v)$: $m$ ordered write/read of object $o$ and value $v$} \\
		\multicolumn{2}{c}{\hl{sync}}
	\end{tabular}
	\vspace{-2.5em}
\end{wrapfigure}

\noindent
The synchronizations further form transitive $\setITHB$ relations
(as described in Section~\ref{sec:c11})
making previously available writes now unavailable to read events.
%
Consider the example \hlref{sync}. As all memory accesses in the
buggy trace $\tau$ are relaxed, 
$\rf{\tau}{W^\rlx(x,1)}{R^\rlx(x,1)}$ does not form a synchronization
between the respective threads. As a result, 
$\nithb{\tau}{W^\rlx(y,1)}{R^\rlx(y,0)}$ and the trace $\tau$ is valid
under \cc. 
%
The buggy trace can be invalidated by introducing weaker fences
$\mathbb{F}^\rel_1$ and $\mathbb{F}^\acq_2$ as shown in \hlref{sync}(ii).
The fences, related as $\sw{\inv{\tau}}{\mathbb{F}^\rel_1}{\mathbb{F}^\acq_2}$
(using \hlref{swFF}), form
$\ithb{\inv{\tau}}{W(y,1)}{Ry(y,0)}$, which violates the coherence of 
$\inv{\tau}$ as $\ithb{\inv{\tau}}{W(y,1)}{Ry(y,0)}$ $\^$ 
$\fr{\inv{\tau}}{Ry(y,0)}{W(y,1)}$. 

To successfully synthesize fences $\mathbb{F}^\rel_1$ and 
$\mathbb{F}^\acq_2$ in \hlref{sync} and similar cases our
technique must recognize candidate $\setSW$ or $\setDOB$
related fence(s) that would form the necessary $\setITHB$ 
and invalidate a buggy trace. In other words, we need to 
recognize the program locations where a desired 
synchronization was absent. 
%
To do so we introduce $\lws$-rules 
(diagrammatically shown in Fig~\ref{fig:so rules}(iii)) 
that essentially capture the absence of a synchronization.

\begin{longtable}{|p{0.11\textwidth} p{0.88\textwidth}|}
	\hline
	\multicolumn{2}{|l|}{\bf \lws-base rule} \\
	
	\hl{ws}: &
	$\forall w^\rel_1 \in \ordwrites{\moge\rel}_\tau$,
	$r^\acq_1 \in \ordreads{\moge\acq}_\tau$,
	$w_2 \in \writes_\tau$ and $r_2 \in \reads_\tau$ \st
	$\seqb{\tau}{r^\acq_1}{r_2}$, $\seqb{\tau}{w_2}{w^\rel_1}$
	if $\fr{\tau}{r_2}{w_2}$ then $\ws{\tau}{r^\acq_1}{w^\rel_1}$ \\
	& ($\setFR$ implies absence of synchronization
		between intermediate $\moge\acq$ and $\moge\rel$ events) \\
		
	& \\
	\multicolumn{2}{|l|}{\bf \lws-rules extended for fences} \\
	
	\hl{wsEF}: & 
	$\forall f^\rel_1 \in \ordfences{\moge\rel}_\tau$,
	$r^\acq_1 \in \ordreads{\moge\acq}_\tau$,
	$w_1 \in \writes_\tau$ and $r_2 \in \reads_\tau$ \st
	$\seqb{\tau}{r^\acq_1}{r_2}$, $\seqb{\tau}{w_1}{f^\rel_1}$
	if $\fr{\tau}{r_2}{w_1}$ then $\ws{\tau}{r^\acq_1}{f^\rel_1}$ \\
	& ($\setFR$ implies absence of synchronization
	between intermediate $\moge\acq$ event and $\moge\rel$ fence) \\
	
	\hl{wsFE}: & 
	$\forall w^\rel_1 \in \ordwrites{\moge\rel}_\tau$,
	$f^\acq_1 \in \ordfences{\moge\acq}_\tau$,
	$w_2 \in \writes_\tau$ and $r_1 \in \reads_\tau$ \st
	$\seqb{\tau}{f^\acq_1}{r_1}$, $\seqb{\tau}{w_2}{w^\rel_1}$
	if $\fr{\tau}{r_1}{w_2}$ then $\ws{\tau}{f^\acq_1}{w^\rel_1}$ \\
	& ($\setFR$ implies absence of synchronization
	between intermediate $\moge\acq$ fence and $\moge\rel$ event) \\
	
	\hl{wsFF}: & 
	$\forall f^\acq_1 \in \ordfences{\moge\acq}_\tau$,
	$f^\rel_2 \in \ordfences{\moge\rel}_\tau$,
	$w_1 \in \writes_\tau$ and $r_1 \in \reads_\tau$ \st
	$\seqb{\tau}{f^\acq_1}{r_1}$, $\seqb{\tau}{w_1}{f^\rel_2}$
	if $\fr{\tau}{r_1}{w_1}$ then $\ws{\tau}{f^\acq_1}{f^\rel_2}$ \\
	& ($\setFR$ implies absence of synchronization
	between intermediate $\moge\acq$ fence and $\moge\rel$ event) \\
	
	\hline
	\multicolumn{2}{r}{\scriptsize where, 
		$\moge\rel$ = $\{\rel,\acqrel,\sc\}$, 
		$\moge\acq$ = $\{\acq,\acqrel,\sc\}$}
\end{longtable}

To summarize, Strategy2 builds on two sets of events:
(i) candidate events that can form a synchronization 
(captured with $\setSW$- and $\setDOB$-rules); and
(ii) candidate events that are originally missing a desired
synchronization (captured with $\setWS$-rules).
If the two sets intersect we get the program locations
for synthesizing fences that would prevent the buggy trace.
We recognize the intersection by a cycle in  
$\ws{\inv{\tau}}{}{}$ $\union$ $\hb{\inv{\tau}}{}{}$ 
(that contains $\sw{\inv{\tau}}{}{}$ and $\dob{\inv{\tau}}{}{}$), 
such as the cycle between $\mathbb{F}^\rel_1$ and 
$\mathbb{F}^\acq_2$ in the example \hlref{sync}.

Note that, 
\snj{sc fences alone ensures optimality inno of fences, str 2 needed for opt in type of fences.}
