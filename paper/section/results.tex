\begin{table}
		\resizebox{\columnwidth}{!}{%
\begin{tabular}{l|r|r|r|r|r|r|r|l}
\multicolumn{1}{c|}{}                       & \multicolumn{1}{c|}{}                                                                             & \multicolumn{1}{c|}{}                                                                            & \multicolumn{1}{c|}{}                                                                                                & \multicolumn{4}{c|}{\textit{Times (in seconds)}}                                                                        & \multicolumn{1}{c}{}                                    \\ \cline{5-8}
\multicolumn{1}{c|}{\multirow{-2}{*}{Name}} & \multicolumn{1}{c|}{\multirow{-2}{*}{\begin{tabular}[c]{@{}c@{}}Buggy\\ executions\end{tabular}}} & \multicolumn{1}{c|}{\multirow{-2}{*}{\begin{tabular}[c]{@{}c@{}}Fences\\ inserted\end{tabular}}} & \multicolumn{1}{c|}{\multirow{-2}{*}{\begin{tabular}[c]{@{}c@{}}Avg no. of\\ instr/trace\end{tabular}}} & \multicolumn{1}{c|}{\cite{cds}} & \multicolumn{1}{c|}{\z} & \multicolumn{1}{c|}{Tool only} & \multicolumn{1}{c|}{Total} & \multicolumn{1}{c}{\multirow{-2}{*}} \\ \hline
dekker\_no\_fence                           & 2                                                                                                 & 2                                                                                                & 19                                                                                                                   & 0.30                            & 0.02                    & {\color[HTML]{00009B} 0.02}    & 0.34                       \\
fib\_mod\_false-unreach-call                & 12                                                                                                & 3                                                                                                & 32                                                                                                                   & 4.44                            & 0.36                    & {\color[HTML]{00009B} 1.93}    & 6.74                       \\
mot\_eg\_modified                           & 292                                                                                               & 4                                                                                                & 40                                                                                                                   & 0.73                            & 0.12                    & {\color[HTML]{00009B} 3.99}    & 4.83                       \\
mot\_eg\_v2\_2                              & 1835                                                                                              & 2                                                                                                & 32                                                                                                                   & 2.56                            & 0.03                    & {\color[HTML]{00009B} 12.18}   & 14.77                      \\
mot\_eg\_v3                                 & 19                                                                                                & 2                                                                                                & 26                                                                                                                   & 0.26                            & 0.02                    & {\color[HTML]{00009B} 0.10}    & 0.37                       \\
mot\_eg\_v3\_modified                       & 29                                                                                                & 5                                                                                                & 27                                                                                                                   & 0.25                            & 0.02                    & {\color[HTML]{00009B} 0.14}    & 0.41                       \\
peterson                                    & 24                                                                                                & 2                                                                                                & 24                                                                                                                   & 0.28                            & 0.03                    & {\color[HTML]{00009B} 0.67}    & 0.98                       \\
read\_write\_lock\_2                        & 524                                                                                               & 4                                                                                                & 41                                                                                                                   & 1.42                            & 6.60                    & {\color[HTML]{00009B} 84.69}   & 92.70                      \\
read\_write\_lock\_unreach\_11              & 1                                                                                                 & 2                                                                                                & 27                                                                                                                   & 0.24                            & 0.02                    & {\color[HTML]{00009B} 0.02}    & 0.28                       \\
read\_write\_lock\_unreach\_12              & 58                                                                                                & 2                                                                                                & 34                                                                                                                   & 0.43                            & 0.02                    & {\color[HTML]{00009B} 0.55}    & 0.99                       \\
read\_write\_lock\_unreach\_13              & 9229                                                                                              & 2                                                                                                & 44                                                                                                                   & 55.51                           & 0.09                    & {\color[HTML]{00009B} 153.14}  & 208.74                     \\
basic                                       & 1                                                                                                 & 2                                                                                                & 21                                                                                                                   & 0.33                            & 0.01                    & {\color[HTML]{00009B} 0.01}    & 0.35                       &                                                         \\
mixed\_eg                                   & 4                                                                                                 & 4                                                                                                & 27                                                                                                                   & 0.33                            & 0.02                    & {\color[HTML]{00009B} 0.06}    & 0.41                       &                                                         \\
dekker\_rlx                                 & 2                                                                                                 & 2                                                                                                & 17                                                                                                                   & 0.32                            & 0.01                    & {\color[HTML]{00009B} 0.35}    & 0.01                       														\\
pgsql0                                      & 7                                                                                                 & 2                                                                                                & 23                                                                                                                   & 0.27                            & 107.13                  & {\color[HTML]{00009B} 603.75}  & 496.35                     \\
publish-sc0                                 & 41                                                                                                & 3                                                                                                & 55                                                                                                                   & 0.36                            & 0.01                    & {\color[HTML]{00009B} 4.60}    & 4.24                       \\
publish-sc1                                 & 41                                                                                                & 3                                                                                                & 55                                                                                                                   & 0.30                            & 0.01                    & {\color[HTML]{00009B} 4.56}    & 4.25                       \\
SB+assert0                                  & 1                                                                                                 & 2                                                                                                & 15                                                                                                                   & 0.32                            & 0.01                    & {\color[HTML]{00009B} 0.34}    & 0.01                       \\
SB+assert1                                  & 1                                                                                                 & 2                                                                                                & 15                                                                                                                   & 0.32                            & 0.01                    & {\color[HTML]{00009B} 0.34}    & 0.01                       \\
thread01                                    & 2                                                                                                 & 2                                                                                                & 13                                                                                                                   & 0.32                            & 0.02                    & {\color[HTML]{00009B} 0.35}    & 0.01                       \\
2+2W0053                                    & 1                                                                                                 & 2                                                                                                & 22                                                                                                                   & 0.33                            & 0.02                    & {\color[HTML]{00009B} 0.36}    & 0.02                       \\
2+2W                                        & 1                                                                                                 & 2                                                                                                & 18                                                                                                                   & 0.32                            & 0.01                    & {\color[HTML]{00009B} 0.35}    & 0.01                       \\
3.2W+lwsync+lwsync+po                       & 1                                                                                                 & 3                                                                                                & 28                                                                                                                   & 0.33                            & 0.02                    & {\color[HTML]{00009B} 0.37}    & 0.02                       \\
DETOUR0928                                  & 1                                                                                                 & 2                                                                                                & 29                                                                                                                   & 0.33                            & 0.02                    & {\color[HTML]{00009B} 0.37}    & 0.02                       \\
m2                                          & 9                                                                                                 & 2                                                                                                & 28                                                                                                                   & 0.33                            & 0.01                    & {\color[HTML]{00009B} 0.47}    & 0.12                       \\
mc                                          & 3                                                                                                 & 2                                                                                                & 39                                                                                                                   & 0.34                            & 0.01                    & {\color[HTML]{00009B} 0.46}    & 0.10                       \\
MOREDETOUR0398                              & 4                                                                                                 & 2                                                                                                & 43                                                                                                                   & 0.45                            & 0.02                    & {\color[HTML]{00009B} 0.64}    & 0.18                       \\
MOREDETOUR0406                              & 3                                                                                                 & 2                                                                                                & 47                                                                                                                   & 0.55                            & 0.02                    & {\color[HTML]{00009B} 0.74}    & 0.17                       \\
MOREDETOUR0685                              & 5                                                                                                 & 2                                                                                                & 48                                                                                                                   & 0.93                            & 0.02                    & {\color[HTML]{00009B} 1.28}    & 0.33                       \\
MOREDETOUR0687                              & 1                                                                                                 & 2                                                                                                & 36                                                                                                                   & 0.34                            & 0.02                    & {\color[HTML]{00009B} 0.40}    & 0.04                       \\
MOREDETOUR0874                              & 2                                                                                                 & 2                                                                                                & 38                                                                                                                   & 0.36                            & 0.03                    & {\color[HTML]{00009B} 0.48}    & 0.10                                                         
\end{tabular}%
}
	\caption{Bare bone results}
	\label{tab:results1}
\end{table}

In order to analyze and evaluate the performance of the algorithm 
proposed in this paper, we created a tool called ----. 77 benchmarks were
executed on it, using a computer with \ishComment{specs}.
Most of these benchmarks have been acquired from verification tool
repositories, namely CDSChecker \cite{cds}, Tracer \cite{tracer2018},
GenMC \cite{genmc-PLDI19}, RCMC \cite{rcmc-POPL18}, Watts \cite{watts}.

All benchmarks can be segregated into two base classes. The first set consists of 
programs whose buggy executions can be stopped by inserting fences. Table \ref{tab:results1}
summarizes these results. The second category of benchmarks contains programs which
either do not provide obtainable results using a model checking tool, or
whose buggy behaviours cannot be stopped simply by using fences. \textit{\{The reason
for this may be because those programs are semantically created to prove
a point instead of being stopped using fences.\}} Benchmarks range from
having a few buggy executions to having thousands of buggy executions. They
also vary in the number of threads and atomic instructions in each of
their traces.

For each concurrent program, we provide the following information as criteria for evaluation:
the number of buggy executions found by the model checker, the number of fences
inserted in the final program, and the average number of atomic instructions
in each trace. The tool is assessed by being timed according to different steps
of the program, including time taken by the model checker, time taken by the
SAT solver and the remaining computation time taken by the tool itself. Programs
were run in two modes- the regular mode with all executions
being taken into account and the \texttt{-t} flag mode, where only the first
trace was taken into consideration for each iteration of the input program.

From the experimental result obtained, it can be noted that a major factor directly affecting 
the tool time is the number of buggy executions. This number affects the tool 
time since each execution is computed individually. Naturally, more executions translates to
greater computation time. This is the case, for example, with benchamarks
\texttt{read\_write\_lock\_unreach\_13} and \texttt{mot\_eg\_v2\_2}, which had thousands
of buggy executions. Computing these were time taking tasks for 
both CDSChecker and the tool itself.

\begin{figure}
	a. $O(n^3)$, \\
	\textit{where n is the number of vertices}\\
	
	
	b. $O((n+e)(c+1))$, \\
	\textit{for n nodes, e edges and c elementary circuits}
	\caption{a. complexity of computing transitive \setHB relations\\
	b. complexity of Johnson's algorithm to compute all elementary cycles in a graph}
\end{figure}

Through experiments, it was also observed that the two most expensive 
operations in the entire process were computing transitive relations for \setHB 
relations, and computing all possible elementary cycles in the graph of \setTO
relations. The metric, ``number of instructions in each trace'' adds
considerably to the former factor. A large number of instructions in each
trace produces a greater number of \setHB relations, thereby resulting in
greater computation time for transitive relations. This is evidenced by the benchmarks
\texttt{publish-sc0} and \texttt{publish-sc1} where the number of instructions in each
execution trace is 55 - a greater than average amount for our benchmarks.
Consequently, the tool takes more time solving them.

This metric does not necessarily apply to \setTO relations.
A greater number of instructions may or may not produce a sizeable \setTO graph.
This depends upon the semantics of the input program. However, chances are,
an intricate graph with many \setTO relations will result in an enormous number of cycles, 
sometimes having exponential numbers, resulting in a much greater
computation time. This is evidenced by looking at the Z3 computation 
time column, since the SAT Solver is only concerned with reducing the cycles
to a minimal required number of fences. For instance, the benchmark \texttt{pgsql0}, though having 
a very small number of buggy executions, has an execution time of a few minutes. Much
of this time in spent on SAT Solving, as can be witnessed. This result reveals 
a very large number of \setTO relations and hence cycles.

\begin{table}
\begin{center}
	% Please add the following required packages to your document preamble:
% \usepackage{multirow}
\begin{tabular}{l|r|r|r|r}
\multicolumn{1}{c|}{Name}                           & \multicolumn{1}{c|}{{\color[HTML]{656565} \begin{tabular}[c]{@{}c@{}}Fences\\ inserted\\ (before)\end{tabular}}} & \multicolumn{1}{c|}{\begin{tabular}[c]{@{}c@{}}Fences\\ inserted\\ (after)\end{tabular}} & \multicolumn{1}{c|}{{\color[HTML]{656565} \begin{tabular}[c]{@{}c@{}}Total\\ time\\ (before)\end{tabular}}} & \multicolumn{1}{c}{\begin{tabular}[c]{@{}c@{}}Total\\ time\\ (after)\end{tabular}} \\ \hline
publish-sc0                                         & {\color[HTML]{656565} 3}                                                                                         & 3                                                                                        & {\color[HTML]{656565} 4.60}                                                                                 & 2.24                                                                               \\
{\color[HTML]{000000} publish-sc1}                  & {\color[HTML]{656565} 3}                                                                                         & {\color[HTML]{000000} 3}                                                                 & {\color[HTML]{656565} 4.56}                                                                                 & {\color[HTML]{000000} 2.56}                                                        \\
mot\_eg\_modified                                   & {\color[HTML]{656565} 4}                                                                                         & 5                                                                                        & {\color[HTML]{656565} 4.83}                                                                                 & 2.47                                                                               \\
{\color[HTML]{000000} fib\_mod\_false-unreach-call} & {\color[HTML]{656565} 3}                                                                                         & {\color[HTML]{000000} 3}                                                                 & {\color[HTML]{656565} 6.74}                                                                                 & {\color[HTML]{000000} 228.93}                                                      \\
peterson                                            & {\color[HTML]{656565} 2}                                                                                         & 3                                                                                        & {\color[HTML]{656565} 0.98}                                                                                 & 0.75                                                                               \\
{\color[HTML]{000000} pgsql0}                       & {\color[HTML]{656565} 2}                                                                                         & {\color[HTML]{000000} 2}                                                                 & {\color[HTML]{656565} 603.75}                                                                               & {\color[HTML]{000000} 0.71}                                                        \\
mot\_eg\_v2\_2                                      & {\color[HTML]{656565} 2}                                                                                         & 2                                                                                        & {\color[HTML]{656565} 14.77}                                                                                & 16.03                                                                              \\
{\color[HTML]{000000} read\_write\_lock\_2}         & {\color[HTML]{656565} 4}                                                                                         & {\color[HTML]{000000} 4}                                                                 & {\color[HTML]{656565} 92.7}                                                                                 & {\color[HTML]{000000} 4.53}                                                        \\
read\_write\_lock\_unreach\_13                      & {\color[HTML]{656565} 2}                                                                                         & 2                                                                                        & {\color[HTML]{656565} 208.74}                                                                               & 200.35                                                                            
\end{tabular}
	\caption{Optimized results}
	\label{tab:results2}
\end{center}
\end{table}

To combat this problem of number of cycles or number of buggy executions going out of bounds,
the optimization discussed in \ref{sec:optis} was applied.
Table \ref{tab:results2} contains a few of these results. Results state that
in most cases, the number of fences remains the same as they were with the 
full optimization, but the time is considerably reduced.

The tool performance cannot be compared with any previous approaches since there have been no
approaches for fence synthesis for the C11 model.