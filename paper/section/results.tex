(tentative timings)
\begin{figure}
	\resizebox{\columnwidth}{!}{%
\begin{tabular}{l|r|r|r|r|r|r|r|l}
\multicolumn{1}{c|}{}                       & \multicolumn{1}{c|}{}                                                                             & \multicolumn{1}{c|}{}                                                                            & \multicolumn{1}{c|}{}                                                                                                & \multicolumn{4}{c|}{\textit{Times (in seconds)}}                                                                        & \multicolumn{1}{c}{}                                    \\ \cline{5-8}
\multicolumn{1}{c|}{\multirow{-2}{*}{Name}} & \multicolumn{1}{c|}{\multirow{-2}{*}{\begin{tabular}[c]{@{}c@{}}Buggy\\ executions\end{tabular}}} & \multicolumn{1}{c|}{\multirow{-2}{*}{\begin{tabular}[c]{@{}c@{}}Fences\\ inserted\end{tabular}}} & \multicolumn{1}{c|}{\multirow{-2}{*}{\begin{tabular}[c]{@{}c@{}}Avg no. of\\ instr/trace\end{tabular}}} & \multicolumn{1}{c|}{\cite{cds}} & \multicolumn{1}{c|}{\z} & \multicolumn{1}{c|}{Tool only} & \multicolumn{1}{c|}{Total} & \multicolumn{1}{c}{\multirow{-2}{*}} \\ \hline
dekker\_no\_fence                           & 2                                                                                                 & 2                                                                                                & 19                                                                                                                   & 0.30                            & 0.02                    & {\color[HTML]{00009B} 0.02}    & 0.34                       \\
fib\_mod\_false-unreach-call                & 12                                                                                                & 3                                                                                                & 32                                                                                                                   & 4.44                            & 0.36                    & {\color[HTML]{00009B} 1.93}    & 6.74                       \\
mot\_eg\_modified                           & 292                                                                                               & 4                                                                                                & 40                                                                                                                   & 0.73                            & 0.12                    & {\color[HTML]{00009B} 3.99}    & 4.83                       \\
mot\_eg\_v2\_2                              & 1835                                                                                              & 2                                                                                                & 32                                                                                                                   & 2.56                            & 0.03                    & {\color[HTML]{00009B} 12.18}   & 14.77                      \\
mot\_eg\_v3                                 & 19                                                                                                & 2                                                                                                & 26                                                                                                                   & 0.26                            & 0.02                    & {\color[HTML]{00009B} 0.10}    & 0.37                       \\
mot\_eg\_v3\_modified                       & 29                                                                                                & 5                                                                                                & 27                                                                                                                   & 0.25                            & 0.02                    & {\color[HTML]{00009B} 0.14}    & 0.41                       \\
peterson                                    & 24                                                                                                & 2                                                                                                & 24                                                                                                                   & 0.28                            & 0.03                    & {\color[HTML]{00009B} 0.67}    & 0.98                       \\
read\_write\_lock\_2                        & 524                                                                                               & 4                                                                                                & 41                                                                                                                   & 1.42                            & 6.60                    & {\color[HTML]{00009B} 84.69}   & 92.70                      \\
read\_write\_lock\_unreach\_11              & 1                                                                                                 & 2                                                                                                & 27                                                                                                                   & 0.24                            & 0.02                    & {\color[HTML]{00009B} 0.02}    & 0.28                       \\
read\_write\_lock\_unreach\_12              & 58                                                                                                & 2                                                                                                & 34                                                                                                                   & 0.43                            & 0.02                    & {\color[HTML]{00009B} 0.55}    & 0.99                       \\
read\_write\_lock\_unreach\_13              & 9229                                                                                              & 2                                                                                                & 44                                                                                                                   & 55.51                           & 0.09                    & {\color[HTML]{00009B} 153.14}  & 208.74                     \\
basic                                       & 1                                                                                                 & 2                                                                                                & 21                                                                                                                   & 0.33                            & 0.01                    & {\color[HTML]{00009B} 0.01}    & 0.35                       &                                                         \\
mixed\_eg                                   & 4                                                                                                 & 4                                                                                                & 27                                                                                                                   & 0.33                            & 0.02                    & {\color[HTML]{00009B} 0.06}    & 0.41                       &                                                         \\
dekker\_rlx                                 & 2                                                                                                 & 2                                                                                                & 17                                                                                                                   & 0.32                            & 0.01                    & {\color[HTML]{00009B} 0.35}    & 0.01                       														\\
pgsql0                                      & 7                                                                                                 & 2                                                                                                & 23                                                                                                                   & 0.27                            & 107.13                  & {\color[HTML]{00009B} 603.75}  & 496.35                     \\
publish-sc0                                 & 41                                                                                                & 3                                                                                                & 55                                                                                                                   & 0.36                            & 0.01                    & {\color[HTML]{00009B} 4.60}    & 4.24                       \\
publish-sc1                                 & 41                                                                                                & 3                                                                                                & 55                                                                                                                   & 0.30                            & 0.01                    & {\color[HTML]{00009B} 4.56}    & 4.25                       \\
SB+assert0                                  & 1                                                                                                 & 2                                                                                                & 15                                                                                                                   & 0.32                            & 0.01                    & {\color[HTML]{00009B} 0.34}    & 0.01                       \\
SB+assert1                                  & 1                                                                                                 & 2                                                                                                & 15                                                                                                                   & 0.32                            & 0.01                    & {\color[HTML]{00009B} 0.34}    & 0.01                       \\
thread01                                    & 2                                                                                                 & 2                                                                                                & 13                                                                                                                   & 0.32                            & 0.02                    & {\color[HTML]{00009B} 0.35}    & 0.01                       \\
2+2W0053                                    & 1                                                                                                 & 2                                                                                                & 22                                                                                                                   & 0.33                            & 0.02                    & {\color[HTML]{00009B} 0.36}    & 0.02                       \\
2+2W                                        & 1                                                                                                 & 2                                                                                                & 18                                                                                                                   & 0.32                            & 0.01                    & {\color[HTML]{00009B} 0.35}    & 0.01                       \\
3.2W+lwsync+lwsync+po                       & 1                                                                                                 & 3                                                                                                & 28                                                                                                                   & 0.33                            & 0.02                    & {\color[HTML]{00009B} 0.37}    & 0.02                       \\
DETOUR0928                                  & 1                                                                                                 & 2                                                                                                & 29                                                                                                                   & 0.33                            & 0.02                    & {\color[HTML]{00009B} 0.37}    & 0.02                       \\
m2                                          & 9                                                                                                 & 2                                                                                                & 28                                                                                                                   & 0.33                            & 0.01                    & {\color[HTML]{00009B} 0.47}    & 0.12                       \\
mc                                          & 3                                                                                                 & 2                                                                                                & 39                                                                                                                   & 0.34                            & 0.01                    & {\color[HTML]{00009B} 0.46}    & 0.10                       \\
MOREDETOUR0398                              & 4                                                                                                 & 2                                                                                                & 43                                                                                                                   & 0.45                            & 0.02                    & {\color[HTML]{00009B} 0.64}    & 0.18                       \\
MOREDETOUR0406                              & 3                                                                                                 & 2                                                                                                & 47                                                                                                                   & 0.55                            & 0.02                    & {\color[HTML]{00009B} 0.74}    & 0.17                       \\
MOREDETOUR0685                              & 5                                                                                                 & 2                                                                                                & 48                                                                                                                   & 0.93                            & 0.02                    & {\color[HTML]{00009B} 1.28}    & 0.33                       \\
MOREDETOUR0687                              & 1                                                                                                 & 2                                                                                                & 36                                                                                                                   & 0.34                            & 0.02                    & {\color[HTML]{00009B} 0.40}    & 0.04                       \\
MOREDETOUR0874                              & 2                                                                                                 & 2                                                                                                & 38                                                                                                                   & 0.36                            & 0.03                    & {\color[HTML]{00009B} 0.48}    & 0.10                                                         
\end{tabular}%
}
	\caption{Results table 1}\label{fig:tabl1}
\end{figure}

Through experiments, it was observed that the two most expensive operations
in the entire process are - computing transitive relations for \setHB's and
computing all possible cycles in the graph of \setTO relations. The algorithm 
to compute all elementary cycles in the directed graph implements
Johnson's algorithm \ref{networkx-cycles}.

\begin{figure}
	a. $O(n^3)$, \\
	\textit{where n is the number of vertices}
	
	
	b. $O((n+e)(c+1))$, \\
	\textit{for n nodes, e edges and c elementary circuits}
	\caption{a. complexity of computing transitive \setHB relations\\
	b. complexity of Johnson's algorithm}
\end{figure}

Table \ref{fig:tabl1} contains a few of the benchmarks used to test this tool.
These benchmarks have been taken from different model checking tools and amended
according to our specifications. At a first glance, it can be noted that
the major factors directly affecting the tool time are- the number of buggy executions
and the number of instructions in each execution.

Firstly, the number of buggy executions affects the tool time since each execution
is being computed and being sent through a series of steps. Hence, more executions means more
computation time. On the other hand, the number of instructions in each trace 
determines the number of relations being formed. A greater number of instructions
results in more \setHB relations. The number of fences to be put between each instruction also
increases and hence, the number of TO relations increases, since all fences are of the
\mosc memory order. Finally, a sizeable graph with many TO relations
results in an exponential number of cycles. 

\begin{figure}
	% Please add the following required packages to your document preamble:
% \usepackage{multirow}
\begin{tabular}{l|r|r|r|r}
\multicolumn{1}{c|}{Name}                           & \multicolumn{1}{c|}{{\color[HTML]{656565} \begin{tabular}[c]{@{}c@{}}Fences\\ inserted\\ (before)\end{tabular}}} & \multicolumn{1}{c|}{\begin{tabular}[c]{@{}c@{}}Fences\\ inserted\\ (after)\end{tabular}} & \multicolumn{1}{c|}{{\color[HTML]{656565} \begin{tabular}[c]{@{}c@{}}Total\\ time\\ (before)\end{tabular}}} & \multicolumn{1}{c}{\begin{tabular}[c]{@{}c@{}}Total\\ time\\ (after)\end{tabular}} \\ \hline
publish-sc0                                         & {\color[HTML]{656565} 3}                                                                                         & 3                                                                                        & {\color[HTML]{656565} 4.60}                                                                                 & 2.24                                                                               \\
{\color[HTML]{000000} publish-sc1}                  & {\color[HTML]{656565} 3}                                                                                         & {\color[HTML]{000000} 3}                                                                 & {\color[HTML]{656565} 4.56}                                                                                 & {\color[HTML]{000000} 2.56}                                                        \\
mot\_eg\_modified                                   & {\color[HTML]{656565} 4}                                                                                         & 5                                                                                        & {\color[HTML]{656565} 4.83}                                                                                 & 2.47                                                                               \\
{\color[HTML]{000000} fib\_mod\_false-unreach-call} & {\color[HTML]{656565} 3}                                                                                         & {\color[HTML]{000000} 3}                                                                 & {\color[HTML]{656565} 6.74}                                                                                 & {\color[HTML]{000000} 228.93}                                                      \\
peterson                                            & {\color[HTML]{656565} 2}                                                                                         & 3                                                                                        & {\color[HTML]{656565} 0.98}                                                                                 & 0.75                                                                               \\
{\color[HTML]{000000} pgsql0}                       & {\color[HTML]{656565} 2}                                                                                         & {\color[HTML]{000000} 2}                                                                 & {\color[HTML]{656565} 603.75}                                                                               & {\color[HTML]{000000} 0.71}                                                        \\
mot\_eg\_v2\_2                                      & {\color[HTML]{656565} 2}                                                                                         & 2                                                                                        & {\color[HTML]{656565} 14.77}                                                                                & 16.03                                                                              \\
{\color[HTML]{000000} read\_write\_lock\_2}         & {\color[HTML]{656565} 4}                                                                                         & {\color[HTML]{000000} 4}                                                                 & {\color[HTML]{656565} 92.7}                                                                                 & {\color[HTML]{000000} 4.53}                                                        \\
read\_write\_lock\_unreach\_13                      & {\color[HTML]{656565} 2}                                                                                         & 2                                                                                        & {\color[HTML]{656565} 208.74}                                                                               & 200.35                                                                            
\end{tabular}
	\caption{Results Table 2}\label{fig:tabl2}
\end{figure}

To combat this problem of number of cycles or number of buggy executions going out of bounds,
an optimization had been introduced and discussed in \ref{sec:optis}. Table \ref{fig:tabl2}
has results of some of the biggest programs in our set with the flag \texttt{-t 1},
which means it looks at only the first trace at a time. Results state that
in most cases, the number of fences remains the same as they were with the 
full optimization. In some of the examples, however, an additional fence or two
is added. The tool time is always considerably lesser when compared with
the original optimization.
