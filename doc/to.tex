\documentclass[dvipsnames]{article}
\setlength{\parindent}{0pt}

\usepackage{amsmath}
\usepackage{amssymb}
\usepackage{smartdiagram}
\usepackage{tikz}
\usetikzlibrary{arrows,positioning}

\usepackage[colorlinks]{hyperref}
\hypersetup{
	colorlinks = true,
	citecolor = {violet},
	linkcolor = {blue},
	urlcolor  = {MidnightBlue}
}

\newcommand{\var}[1]{\color{OliveGreen} \texttt{#1}\color{black}}
\newcommand{\fun}[2]{\color{Sepia}\texttt{#1(\color{Gray}\textit{#2}\color{Sepia})}\color{black}}
\newcommand{\varinfo}[1]{\scriptsize \texttt{#1} \normalsize}
\newcommand{\class}[1]{\color{DarkOrchid}\texttt{#1}\color{black}}

\newcommand{\rf}[2]{{#1} {\color{Blue}$\rightarrow^{rf}$} \color{black}{#2}}
\newcommand{\mo}[2]{{#1} {\color{Blue}$\rightarrow^{mo}$} \color{black}{#2}}
\newcommand{\tor}[2]{{#1} {\color{OliveGreen}$\rightarrow^{to}$} \color{black}{#2}}

\date{}
\begin{document}

\title{to.py}
\maketitle

Total order (TO) is a binary relation on some set, which is antisymmetric, transitive, and a connex relation. Formally, a binary relation $\leq$  is a total order on a set X if the following statements hold for all a, b and c in X:
\begin{enumerate}
    \item \textbf{Antisymmetry}\\
        if a $\leq$ b and b $\leq$ a, then a = b
    \item \textbf{Transitivity}\\
        if a $\leq$ b and b $\leq$ c then a $\leq$ c
    \item \textbf{Connexity}\\
        a $\leq$ b or b $\leq$ a
\end{enumerate}

For C/C++11, these rules translate to the following:
\tikzset{
    %Define standard arrow tip
    >=stealth',
    %Define style for boxes
    punkt/.style={
           rectangle,
           minimum height=2em,
           text centered},
}

\begin{enumerate}
\setcounter{enumi}{-1}
\item
	For a given variable, if there is a read from relation between a read and a write instruction, then the SC instructions above the write are in TO with the SC instructions below the read.\\
	\begin{tikzpicture}[baseline = (current bounding box.north)]
    \node[punkt] (x) {X:$F_{sc}$};
    \node[punkt,below=of x] (a) {A:$Wx$}
        edge[-](x);
    \node[punkt,right=of x] (b) {B:$Rx$};
    \node[punkt, below=of b] (y) {Y:$F_{sc}$}
    	edge[-](b);
  	\draw [->] (a) -> (b) node[midway,above] {rf};
    \end{tikzpicture}\\
    \rf{A}{B} $\implies$ \too{X}{Y}

\item
	\begin{enumerate}[a.]
	\item
		For a given variable, for an SC read instruction reading from an SC write instruction, there is a TO relation between the read and the write instructions.\\
		\rf{A:$W_{sc}$}{B:$R_{sc}$}\\
		\too{A}{B}
	
	\item
		For a given variable,, for an SC read instruction reading from a write instruction \textit{M}, the read instruction is in TO with all SC write instruction which happen after \textit{M}.\\
		A:$W_{sc}x$, M:$Wx$, B:$R_{sc}x$\\
		\hb{M}{A} $\land$ \rf{M}{B} $\implies$ \too{B}{A}
	\end{enumerate}

\item
	For a given variable, all SC instructions above a read instruction reading from a write instruction \textit{$M_1$} are in TO with all SC write instruction modifying the variable after \textit{$M_1$}.
	\begin{tikzpicture}[baseline = (current bounding box.north)]
	\node[punkt](x) {X:$F_{sc}$};
	\node[punkt, below=of x](b) {B:$Rx$}
		edge[-](x);
	\node[punkt, right=of b](m1) {M$_1:Wx$};
	\node[punkt, below=of m1](m2) {M$_2:W_{sc}$};
  	\draw [->] (m1) -> (b) node[midway,above] {rf};
  	\draw [->] (m1) -> (m2) node[midway,right] {mo};
  	\end{tikzpicture}\\
  	\too{X}{M$_2$}
 
\item
	\begin{tikzpicture}[baseline = (current bounding box.north)]
	\node[punkt](a) {A:$Wx$};
	\node[punkt, below=of a](x) {X:$F_{sc}$}
		edge[-](a);
	\node[punkt, right=of a](y) {Y:$F_{sc}$};
	\node[punkt, below=of y](b) {B:$Rx$}
		edge[-](y);
	\node[punkt, right=of y](m) {M:$Wx$};
	\end{tikzpicture}\\
	\rf{M}{B} $\land$ \mo{M}{A} $\implies$ \too{Y}{X}

\item
	\begin{tikzpicture}[baseline = (current bounding box.north)]
	\node[punkt](a) {A:$Wx$};
	\node[punkt, below=of a](x) {X:$F_{sc}$}
		edge[-](a);
	\node[punkt, right=of a](y) {Y:$F_{sc}$};
	\node[punkt, below=of y](b) {B:$Wx$}
		edge[-](y);
	\end{tikzpicture}\\
	\begin{enumerate}[a.]
	\item if B:$W_{sc}x$\\
		\mo{B}{A} $\implies$ \too{B}{X}
	
	\item if A:$W_{sc}x$\\
		\mo{B}{A} $\implies$ \too{Y}{A}
	
	\item
		\mo{B}{A} $\implies$ \too{Y}{X}
	\end{enumerate}
	
\end{enumerate}

\section{Finding TO edges between fences}
Each rule has a function that runs according to it and determines TO relations between fences. After immediate relations are calculated, transitive TO relations between the fences are also calculated in function \fun{all\_to}{}.

\end{document}